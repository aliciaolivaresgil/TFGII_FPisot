\capitulo{3}{Conceptos teóricos}

En este capítulo se llevará a cabo una descripción de los conceptos necesarios para comprender el funcionamiento de la aplicación desarrollada. Es necesario tener un conocimiento acerca de los modelos del lenguaje que proporcionan el origen de datos en los que se basan los \acrfull{poi}.



\section{Large Language Models (LLM)}

Empezamos por el concepto más general para luego ir acercándonos a la parte más concreta del desarrollo de la aplicación. Los modelos de lenguaje a gran escala es un tipo de inteligencia artificial que ha sido entrenado para comprender \acrfull{nlp} que es la manera en que se comunican las personas. Estas inteligencias artificiales son entrenadas entonces con ingentes cantidades de datos que los hacen capaces de comprender peticiones, responder a las mismas en los mismos términos de lenguaje generando una especie de comunicación entre el usuario y la máquina.



\subsubsection{Uso de LLMs en la aplicación de este TFG}

Y subsecciones. 


\section{Retrieval-Augmented Generation (RAG)}
Generación Aumentada por Recuperación es una técnica usada los modelos de inteligencia artificial en la cual se obtiene información para nutrir a un modelo de gran escala que ya ha sido entrenado, de esta manera amplía su conocimiento y es capaz de generar una respuesta más precisa, actualizada y completa. Para ello, selecciona la información más afín de los datos aportados que se le han facilitado con la información requerida por la entrada del modelo. Veamos cómo lo consigue:
\begin{enumerate}
	\item \textbf{Vectorización}: también llamada embeddings: consiste en transformar la información facilitada y representarla en vectores de n dimensiones.
	\item segundo item.
\end{enumerate}

\subsubsection{Uso de RAGs en la aplicación de este TFG}



\section{Agentes}
\label{sec:agentes}
Pese a que los LLM como se ha visto en su descripción son herramientas muy poderosas que han cambiado la visión del mundo en todos sus campos productivos, sufren de una limitación: el momento en el que terminan su entrenamiento los modelos y son puestos en marcha para el público supone una brecha de entrada de conocimiento del mundo que le rodea y por tanto si una pieza vital de información es requerida en dicho periodo de tiempo, el modelo puede alucinar o no dar con una salida satisfactoria. Para cubrir este hueco de información surgen los Agentes.
Estos elementos pueden ser vistos como aplicaciones que alimentarán al modelo con un conjunto de herramientas tales como motores de búsqueda, bases de datos, páginas web, etc. Una vez provisto con esta información el modelo es capaz de razonar acerca de las acciones que debe cumplir para obtener el mejor resultado.


\section{Imágenes}

Se pueden incluir imágenes con los comandos standard de \LaTeX, pero esta plantilla dispone de comandos propios como por ejemplo el siguiente:

\imagen{escudoInfor}{Autómata para una expresión vacía}{.5}

Las referencias se incluyen en el texto usando cite~\cite{wiki:latex}. Para citar webs, artículos o libros~\cite{koza92}, si se desean citar más de uno en el mismo lugar~\cite{bortolot2005, koza92}.

\section{Listas de items}

Existen tres posibilidades:

\begin{itemize}
	\item primer item.
	\item segundo item.
\end{itemize}

\begin{enumerate}
	\item primer item.
	\item segundo item.
\end{enumerate}

\begin{description}
	\item[Primer item] más información sobre el primer item.
	\item[Segundo item] más información sobre el segundo item.
\end{description}
	
\begin{itemize}
\item 
\end{itemize}

\section{Tablas}

Igualmente se pueden usar los comandos específicos de \LaTeX o bien usar alguno de los comandos de la plantilla.

\tablaSmall{Herramientas y tecnologías utilizadas en cada parte del proyecto}{l c c c c}{herramientasportipodeuso}
{ \multicolumn{1}{l}{Herramientas} & App AngularJS & API REST & BD & Memoria \\}{ 
HTML5 & X & & &\\
CSS3 & X & & &\\
BOOTSTRAP & X & & &\\
JavaScript & X & & &\\
AngularJS & X & & &\\
Bower & X & & &\\
PHP & & X & &\\
Karma + Jasmine & X & & &\\
Slim framework & & X & &\\
Idiorm & & X & &\\
Composer & & X & &\\
JSON & X & X & &\\
PhpStorm & X & X & &\\
MySQL & & & X &\\
PhpMyAdmin & & & X &\\
Git + BitBucket & X & X & X & X\\
Mik\TeX{} & & & & X\\
\TeX{}Maker & & & & X\\
Astah & & & & X\\
Balsamiq Mockups & X & & &\\
VersionOne & X & X & X & X\\
} 
Algunos conceptos teóricos de \LaTeX{} \footnote{Créditos a los proyectos de Álvaro López Cantero: Configurador de Presupuestos y Roberto Izquierdo Amo: PLQuiz}.