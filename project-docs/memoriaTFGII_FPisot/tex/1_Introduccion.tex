\capitulo{1}{Introducción}

En la era de la digitalización, la movilidad urbana y el turismo sostenible se han
convertido en pilares fundamentales para la planificación y desarrollo de las ciudades
inteligentes. Las aplicaciones móviles, impulsadas por tecnologías avanzadas y datos
abiertos, están desempeñando un papel crucial en la promoción de estilos de vida
saludables y sostenibles. Este proyecto se centra en el desarrollo de una aplicación móvil
que genera rutas saludables para ciclistas y peatones, utilizando un framework de datos
abiertos GIS (Sistema de Información Geográfica) y modelos de lenguaje (LLM), en el
contexto de la promoción de ciudades inteligentes, la Agenda Urbana y los Objetivos de
Desarrollo Sostenible (ODS). 
