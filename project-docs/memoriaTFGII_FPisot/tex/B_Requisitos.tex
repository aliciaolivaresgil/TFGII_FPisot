\apendice{Especificación de Requisitos}

\section{Introducción}

Una muestra de cómo podría ser una tabla de casos de uso:

% Caso de Uso 1 -> Consultar Experimentos.
\begin{table}[p]
	\centering
	\begin{tabularx}{\linewidth}{ p{0.21\columnwidth} p{0.71\columnwidth} }
		\toprule
		\textbf{CU-1}    & \textbf{Activación de permiso de localización \acrshort{gps} del dispositivo}\\
		\toprule
		\textbf{Versión}              & 1.0    \\
		\textbf{Autor}                & \autor \\
		\textbf{Requisitos asociados} & RF-1 \\
		\textbf{Descripción}          & Concede permisos de localización al dispositivo \\
		\textbf{Precondición}         & Acceder a la aplicación \\
		\textbf{Acciones}             &
		\begin{enumerate}
			\def\labelenumi{\arabic{enumi}.}
			\tightlist
			\item El usuario abre la aplicación o ha restringido los accesos a la localización
			\item Un mensaje del \acrfull{so} solicita conceder permisos
		\end{enumerate}\\
		\textbf{Postcondición}        & El permiso de seguimiento del usuario ha sido concedido \\
		\textbf{Excepciones}          & El usuario no concede los permisos \\
		\textbf{Importancia}          & Alta  \\
		\bottomrule
	\end{tabularx}
	\caption{CU-1 Activación de permiso de localización \acrshort{gps} del dispositivo.}
\end{table}

\begin{table}[p]
	\centering
	\begin{tabularx}{\linewidth}{ p{0.21\columnwidth} p{0.71\columnwidth} }
		\toprule
		\textbf{CU-1}    & \textbf{Activación de sensor \acrshort{gps} en el dispositivo}\\
		\toprule
		\textbf{Versión}              & 1.0    \\
		\textbf{Autor}                & \autor \\
		\textbf{Requisitos asociados} & RF-2 \\
		\textbf{Descripción}          & Activa el uso de GPS en el dispositivo \\
		\textbf{Precondición}         & El dispositivo debe tener un sensor \acrshort{gps}. \\
		\textbf{Acciones}             &
		\begin{enumerate}
			\def\labelenumi{\arabic{enumi}.}
			\tightlist
			\item El usuario activará el sensor de \acrshort{gps} de su dispositivo a través de cualquier de cualquier menú o atajo de configuración.
		\end{enumerate}\\
		\textbf{Postcondición}        & La aplicación tiene acceso a la posición del dispositivo. \\
		\textbf{Excepciones}          & Problemas con el sensor \acrshort{gps} del dispositivo. \\
		\textbf{Importancia}          & Alta \\
		\bottomrule
	\end{tabularx}
	\caption{CU-2 Activación de sensor \acrshort{gps} en el dispositivo.}
\end{table}

\begin{table}[p]
	\centering
	\begin{tabularx}{\linewidth}{ p{0.21\columnwidth} p{0.71\columnwidth} }
		\toprule
		\textbf{CU-1}    & \textbf{Ejemplo de caso de uso}\\
		\toprule
		\textbf{Versión}              & 1.0    \\
		\textbf{Autor}                & Alumno \\
		\textbf{Requisitos asociados} & RF-xx, RF-xx \\
		\textbf{Descripción}          & La descripción del CU \\
		\textbf{Precondición}         & Precondiciones (podría haber más de una) \\
		\textbf{Acciones}             &
		\begin{enumerate}
			\def\labelenumi{\arabic{enumi}.}
			\tightlist
			\item Pasos del CU
			\item Pasos del CU (añadir tantos como sean necesarios)
		\end{enumerate}\\
		\textbf{Postcondición}        & Postcondiciones (podría haber más de una) \\
		\textbf{Excepciones}          & Excepciones \\
		\textbf{Importancia}          & Alta o Media o Baja... \\
		\bottomrule
	\end{tabularx}
	\caption{CU-1 Nombre del caso de uso.}
\end{table}

\section{Objetivos generales}
La misión fundamental de este proyecto persigue conseguir los siguientes propósitos:
\begin{itemize}
	\item Desarrollar una aplicación útil para el usuario que consiga proporcionar información valiosa a la hora de realizar turismo sostenible en una ciudad.
	\
\end{itemize}
\section{Catálogo de requisitos}
\subsection{Requisitos funcionales}
\begin{itemize}
	\item \textbf{RF-1 Solicitar permisos de uso de GPS:} Eco City Tours utiliza la posición del dispositivo y el usuario debe garantizar el permiso de uso.
	\item \textbf{RF-2 Solicitud de activación de GPS:} Eco City Tours debe detectar que se ha desactivado el GPS y solicitar su activación para el uso de la aplicación.
	\item \textbf{RF-3 Activación/Desactivación de seguimiento de usuario:} se debe mostrar el camino seguido por el usuario sobre el mapa siempre que así lo indique, dicho seguimiento se puede activar y desactivar a petición en cualquier momento.
	\item \textbf{RF-4 Centrar la situación actual del usuario sobre mapa:} se debe mover la vista del mapa a la ubicación actual del usuario a petición del mismo. También se puede dejar fija la ubicación del usuario en el centro del mapa.
	\item \textbf{RF-5 Selección de Tour:} El usuario rellenará sus preferencias en una pantalla formulario para indicar qué ciudad quiere visitar, los puntos de interés que quiere o sus gustos al viajar.
	\item \textbf{RF-6 Cálculo de información a través de un \acrfull{llm}}: la aplicación debe proporcionar información de \acrfull{pdi} según los criterios introducidos.
	\item \textbf{RF-7 Eliminación de \acrshort{pdi}}: el usuario podrá eliminar los lugares que no quiera visitar lo que recalculará la ruta óptima de nuevo.
	\item \textbf{RF-8 Cálculo de ruta optimizada}: el Eco City Tour que una los \acrshort{pdi} tiene que seguir una ruta adaptada al medio de transporte elegido por el usuario siendo el más corto posible.
	\item \textbf{RF-9 Capacidad de añadir un \acrshort{pdi}:} el usuario podrá introducir el texto de un lugar que conozca para agregar dicho lugar a su ruta.
	\item \textbf{RF-10 Unirse a Eco City Tour:} el usuario podrá unirse a la ruta calculada cuando así lo requiera, calculando para ello una ruta que lo una de la manera más corta posible a dicho tour.
	
	
\end{itemize}
\subsection{Requisitos no funcionales}
\begin{itemize}
	\item \textbf{RNF-1 Rendimiento:} la aplicación debe ...  
	\item \textbf{RNF-2 Usabilidad:} Eco City Tours debe ser intuitiva y fácil de entender y utilizar.  
	\item \textbf{RNF-4 Disponibilidad:} la aplicación debe estar disponible el mayor tiempo posible.
	\item \textbf{RNF-5 Mantenibilidad:} la aplicación debe ser fácilmente modificable.
\end{itemize}
\section{Especificación de requisitos}


