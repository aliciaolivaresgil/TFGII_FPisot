\capitulo{1}{Introducción}


El crecimiento de población en las ciudades \cite{nieuwenhuijsen_urban_2020} y el turismo como catalizador de la gentrificación supone el gran campo de batalla para gobiernos locales de los países occidentales que han visto como la falta de una legislación controlada del turismo supone un grave problema afectando múltiples niveles de la convivencia, economía y el medio ambiente. A pesar de los avances en la promoción de un nuevo modelo urbano, muchas urbes aún enfrentan desafíos significativos en la integración de prácticas sostenibles en la vida cotidiana de sus habitantes. La falta de información accesible y personalizada sobre rutas y actividades que promuevan la movilidad sostenible y el turismo responsable es un marco común que se debe desarrollar si se quiere evitar que el conflicto crezca sin fin. Esta brecha de información impide que tanto residentes como turistas adopten hábitos más sostenibles que beneficien a la comunidad local y al medio ambiente en un marco global.

Fomentar el Turismo Sostenible fomentando el \acrfull{ods11} supone una gran oportunidad para intentar contrarrestar la deriva actual. Y es que el turismo es un motor fundamental de la economía a nivel global y por tanto tiene la capacidad de transformarse para ayudar a la sostenibilidad del planeta como bien recoge Vaid \cite{vaid_sustainable_2024-1} indicando metas para que el turismo facilite la consecución de los \acrfull{ods}. Destaca en este marco de trabajo el \acrshort{ods11}, que se centra en hacer que las ciudades y los asentamientos humanos sean sostenibles. Según el informe de la \textit{UNESCO} sobre el ODS 11 \cite{ionescu_progress_2024}, este objetivo no solo es crucial por sí mismo, sino que actúa como un multiplicador, influyendo indirectamente en la consecución de otros \acrshort{ods} debido a su enfoque integral y transversal.

En este contexto, la aplicación móvil que proponemos se alinea con estos esfuerzos al proporcionar una herramienta práctica y accesible para la promoción del \acrfull{ods11} y la movilidad sostenible. La aplicación producto de este \acrshort{tfg}, desarrollada en Flutter, utiliza modelos de lenguaje a gran escala (\acrfull{llm}) y el marco de trabajo \textbf{LangChain} para generar rutas turísticas personalizadas que conecten puntos de interés (\acrfull{poi}) que se visualizan mediante herramientas de código abierto como \acrfull{osm}. La aplicación se enfoca en las preferencias del usuario, ofreciendo rutas optimizadas para ciclistas y peatones promoviendo así la movilidad sostenible. Al integrar datos y tecnología avanzada, esta solución no solo facilita una experiencia turística enriquecedora, sino que también fomenta prácticas sostenibles como la deslocalización del turismo que lejos de suponer un impacto negativo para los turistas \cite{mitas_tell_2023} puede también sostener la forma de vida las comunidades locales, \textbf{teniendo así un impacto positivo en el ámbito local y en medio ambiente global.}


\end{document}