\capitulo{1}{Introducción}


La creciente urbanización y sus desafíos asociados han destacado la necesidad de desarrollar ciudades más sostenibles e inclusivas. A pesar de los avances en esta promoción, muchas urbes aún enfrentan desafíos significativos en la integración de prácticas sostenibles en la vida cotidiana de sus habitantes. La falta de información accesible y personalizada sobre rutas y actividades que promuevan la movilidad sostenible y el turismo responsable es un problema común. Esta brecha de información impide que tanto residentes como turistas adopten hábitos más sostenibles que beneficien al medio ambiente y a la comunidad local.


Para abordar estos desafíos, se han desarrollado diversas aplicaciones y plataformas que buscan promover la sostenibilidad urbana a través de la tecnología. Por ejemplo, proyectos educativos han integrado los ODS en el aprendizaje basado en problemas universitarios, demostrando que este enfoque no solo facilita la educación sobre los ODS, sino que también motiva a los estudiantes a desarrollar soluciones innovadoras para problemas reales\footnote{Markiegi, U., \& Aldalur, I. (2024). Abordando los Objetivos de Desarrollo Sostenible en el Aprendizaje Basado en Problemas universitario. En Actas de las JENUI - Vol. 9 (pp. 251-254). La Coruña.}\footnote{Markiegi, U., Aldalur, I., \& Perez, A. (2023). Integrando los ODS en el grado de Ingeniería Informática. En Actas de las JENUI - Vol. 8 (pp. 249-256). Granada.}. Estos esfuerzos educativos subrayan la importancia de hacer que los ODS sean más conocidos y comprendidos, fomentando una generación de futuros profesionales comprometidos con la sostenibilidad.


En este contexto, la aplicación móvil que proponemos se alinea con estos esfuerzos al proporcionar una herramienta práctica y accesible para la promoción del ODS 11 y la movilidad sostenible.

En este contexto, el Objetivo de Desarrollo Sostenible (ODS) 11, que se centra en hacer que las ciudades y los asentamientos humanos sean inclusivos, seguros, resilientes y sostenibles, emerge como un pilar fundamental. Según el informe de síntesis de la UNESCO sobre el ODS 11, este objetivo no solo es crucial por sí mismo, sino que actúa como un multiplicador, influyendo indirectamente en la consecución de otros ODS debido a su enfoque integral y transversal\footnote{\href{https://uis.unesco.org/sites/default/files/documents/sdg_11_synthesis_report_2023_v11_0_4.pdf}{UNESCO. (2023). SDG 11 Synthesis Report.}}.

Nuestra aplicación, desarrollada con Flutter, utiliza modelos de lenguaje de gran escala (\acrfull{llm}) y el framework LangChain para generar rutas turísticas personalizadas. Estas rutas conectan puntos de interés (\acrfull{poi}) y se visualizan mediante herramientas open-source como \acrfull{osm}.

La aplicación se enfoca en las preferencias del usuario, ofreciendo rutas optimizadas para ciclistas y peatones, y promoviendo así la movilidad sostenible. Al integrar datos y tecnología avanzada, nuestra solución no solo facilita una experiencia turística enriquecedora, sino que también fomenta prácticas sostenibles que pueden tener un impacto positivo en la comunidad y el medio ambiente.

\end{document}