\capitulo{5}{Aspectos relevantes del desarrollo del proyecto}

En este apartado se justificarán las decisiones tomadas acerca del desarrollo de la aplicación, tanto en la construcción de los prototipos como en la generación de la aplicación móvil y sus herramientas y documentación.

Este apartado pretende recoger los aspectos más interesantes del desarrollo del proyecto, comentados por los autores del mismo.
Debe incluir desde la exposición del ciclo de vida utilizado, hasta los detalles de mayor relevancia de las fases de análisis, diseño e implementación.
Se busca que no sea una mera operación de copiar y pegar diagramas y extractos del código fuente, sino que realmente se justifiquen los caminos de solución que se han tomado, especialmente aquellos que no sean triviales.
Puede ser el lugar más adecuado para documentar los aspectos más interesantes del diseño y de la implementación, con un mayor hincapié en aspectos tales como el tipo de arquitectura elegido, los índices de las tablas de la base de datos, normalización y desnormalización, distribución en ficheros3, reglas de negocio dentro de las bases de datos (EDVHV GH GDWRV DFWLYDV), aspectos de desarrollo relacionados con el WWW...
Este apartado, debe convertirse en el resumen de la experiencia práctica del proyecto, y por sí mismo justifica que la memoria se convierta en un documento útil, fuente de referencia para los autores, los tutores y futuros alumnos.

\section{Elección de agentes}
A la hora de obtener información que procesar por el método \acrshort{rag} se valoraron muchas fuentes de datos. Uno de los origenes de datos de información turística favoritos de los usuarios es Tripadvisor. Contar con la información actualizada de este gigante turístico suponía un gran aliciente. Sin embargo se desestimó su uso por varias razones: la información se podía obtener a través del método scraping o webscraping que toma la información en bruto de la página web y se podía postprocesar, sin embargo, dicha práctica incumpliría los Terminos de Uso del Servicio, ya que Tripadvisor usa un acceso a través de API para obtener la información de su base de datos. Una vez más y cómo ya pasara con ciertos servicios de Google, son gratuitos en un principio pero depende de su uso. En primer lugar se necesita una forma de pago para poder empezar a usar el servicio y su utilización si sobrepasa las 5.000 peticiones al mes incurriría en gastos al desarrollador. En este caso y como pasara con Google y OSM se buscó alternativas que funcionaran de manera análoga a Tripadvisor para nutrir el \acrshort{rag}.