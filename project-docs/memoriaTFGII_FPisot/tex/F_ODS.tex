\apendice{Anexo de sostenibilización curricular}

\section{F.1. Introducción}
Durante el desarrollo del Trabajo Final de Carrera, he utilizado e implementado las disciplinas de la sostenibilidad en el diseño de una aplicación móvil. Ha sido de hecho un gran reto el tratar de generar rutas turísticas de tal manera que promocionen la sostenibilidad. Al final la solución presentada cumple con el \acrfull{ods11} fomentando principalmente la sostenibilidad de ciudades y los asentamientos humanos.

El valor principal de este \acrfull{ods}, como ya se ha tratado, radica en su papel de facilitador en la medida que, al resolver de manera integral y transversal todas las cuestiones relacionadas con la sostenibilidad de las ciudades, tiene un efecto secundario positivo en los demás \acrshort{ods}.

\section{Competencias de Sostenibilidad Adquiridas}
A lo largo del trabajo de este proyecto, he seleccionado y elaborado las competencias de sostenibilidad en las siguientes áreas:
	
	\subsection{Contextualización Crítica del Conocimiento}
	Puedo contextualizar el conocimiento adquirido y relacionarlo críticamente con los desafíos globales y locales en las áreas sociales, económicas y ambientales tan influenciadas por el turismo sostenible.
		
	\subsection{Uso Sostenible de Recursos}
	El desarrollo de la aplicación se enfocó en la utilización sostenible de los recursos. Traté de aplicar la eficiencia en el desarrollo de software al usar tecnologías que consumen la menor energía posible y al aprovechar el uso de herramientas open-source que son fácilmente accesibles, por ejemplo \acrfull{osm} .
		
	\subsection{Participación en Procesos Comunitarios}
	El desarrollo de esta aplicación exige una comprensión en profundidad de los procesos comunitarios. Se ha llevado a cabo por ejemplo al decir que no solo los turistas se benefician de su uso, sino que las rutas turísticas también promueven la movilidad sostenible y mejoran el bienestar de la comunidad local.
			
	\subsection{Principios Éticos y Valores de Sostenibilidad}
	 He incorporado prácticas éticas y valores de sostenibilidad en todo el trabajo descartando por ejemplo aquellas modificaciones que pudieran dar como resultado un conflicto entre la comunidad local y la turística.
				
\section{Aplicación de Competencias en el Proyecto}
				
Se utilizaron las siguientes competencias aprendidas en el desarrollo del proyecto:
	\subsection{Diseño y Funcionalidad de la Aplicación}
						
	El diseño de la aplicación ha seguido un enfoque centrado en el usuario cuyo propósito era diseñar una solución intuitiva, fácil de usar que promueva la movilidad sostenible: con ciclistas o peatones que disfruten de un viaje personalizado, uniendo los \acrfull{poi} de una manera eficiente usando planificadores de rutas optimizados a tal efecto.
						
	\subsection{Impacto Social y Ambiental}
	La realización del proyecto busca no solo ser ventajosa para los turistas, sino que también reduce las emisiones de carbono al tiempo que permite a los miembros de la comunidad contribuir a la sostenibilidad de dicha ciudad en su conjunto.
							
	\subsection{Educación y Conciencia ambiental}
	El proyecto informará a los ciudadanos no solo con datos de rutas turísticas específicas, sino también buscará sensibilizar acerca de la movilidad sostenible y los beneficios ambientales que provienen de los dilemas referentes al transporte en los que participan.
								
\section {Conclusión}
Este Trabajo ha sido una experiencia informativa y esclarecedora para mí, tal y como indica en el artículo \cite{markiegi}, trabajar en el desarrollo de una aplicación que pone en práctica soluciones que favorecen los ODS te hace más consciente de los múltiples factores que impiden su cumplimiento. Este tiempo me ha dotado del enfoque que se necesita para enfrentar, de manera consciente y sostenible, los desafíos actuales y futuros con una visión global que trabaje activamente hacia un desarrollo sostenible y me siento capaz de liderar nuevas soluciones tecnológicas que potencien los ODS.

