\apendice{Especificación de diseño}

\section{Introducción}
En este anexo, se detallan las especificaciones de diseño del proyecto, enfocadas en los aspectos fundamentales para el desarrollo de la aplicación. Se describen cómo se organizan los datos que lo componen, el diseño arquitectónico, y el diseño procedimental de los objetos definidos. Estas especificaciones son clave para asegurar la comprensión y la implementación de cada uno de los elementos que componen \textit{Eco City Tours}.
\section{Diseño de datos}

En esta sección se presenta el diseño de datos del sistema a un alto nivel de abstracción. Para explicar dicho diseño se usarán varios diagramas donde se muestran las dos principales entidades del sistema y las relaciones entre ellas, centrándose en su estructura y atributos relevantes. 

\imagen{diagrama-entidades}{Diagrama de entidades. Eco City Tour y \acrlong{pdi}}
\imagen{diagrama-relacional}{Diagrama relacional.}
A continuación, se describen las entidades, sus atributos y las relaciones que existen entre ellas.
\subsection{Entidades del Sistema}
\begin{itemize}
	\item \textbf{EcoCityTour}: Representa una ruta turística generada por la aplicación. Incluye la información general sobre el recorrido.
	\begin{itemize}
		\item \textbf{city}: Ciudad, pueblo o lugar en el que se realiza la ruta.
		\item \textbf{mode}: Medio de transporte elegido por el usuario (a pie o bicicleta).
		\item \textbf{duration}: Duración estimada del recorrido.
		\item \textbf{distance}: Distancia total de la ruta.
		\item \textbf{userPreferences}: Preferencias del usuario (naturaleza, museos, etc.).
	\end{itemize}
	
	\item {\textbf{PointOfInterest}}: Representa un \acrlong{pdi} dentro de la ruta.
	\begin{itemize}
		\item \textbf{name}: Nombre del lugar.
		\item \textbf{gps}: Coordenadas GPS del sitio (latitud y longitud).
		\item \textbf{description}: Breve descripción del lugar.
		\item \textbf{url}: Enlace a información adicional sobre el lugar.
		\item \textbf{imageUrl}: URL de la imagen representativa del lugar.
		\item \textbf{rating}: Calificación promedio del punto de interés.
	\end{itemize}
\end{itemize}

\subsection{Relaciones entre Entidades}
\begin{itemize}
	\item \textbf{EcoCityTour -- PointOfInterest}: Una relación  donde:
	\begin{itemize}
		\item Un \textbf{EcoCityTour} posee \textbf{0..n} puntos de interés (\textbf{PointOfInterest}).
		\item Cada \textbf{PointOfInterest} pertenece a exactamente un \textbf{EcoCityTour}.
	\end{itemize}
\end{itemize}


\section{Diseño procedimental}
En esta sección se explicará el flujo de trabajo relacionado con las tareas más importantes de la aplicación, asociando cada proceso a sus respectivos \textbf{Casos de Uso (CU)} descritos en la subsección~\ref{subsec:casos-uso}.

\subsection{Carga de \textit{EcoCityTour} desde pantalla de selección de preferencias (CU01.1)}
La generación de un nuevo tour turístico una vez que el usuario ha definido sus preferencias es el proceso fundamental que se muestra en la Figura~\ref{fig:secuence}, correspondiente al \textbf{CU01.1 Calcular Eco City Tour}.

El proceso comienza con el evento \texttt{loadTourEvent()}, disparado por el usuario desde la pantalla de selección (\textit{TourSelectionScreen}). Este evento desencadena la siguiente secuencia de interacciones:
\begin{enumerate}
	\item El \textit{TourBloc} inicia la generación del tour solicitando a \textit{ServicioLLM} una lista inicial de \acrlong{pdi} basados en las preferencias del usuario.
	\item \textit{ServicioLLM} devuelve un conjunto de \acrlong{pdi}, que luego se enriquecen con información adicional obtenida a través de \textit{Google Places} mediante el método \texttt{searchPlacePOI}.
	\item La información enriquecida de los \acrlong{pdi} es enviada a \textit{OptimizationService}, que calcula una ruta optimizada utilizando criterios de sostenibilidad y las preferencias del usuario. Esto se realiza a través del método \texttt{optimizedRoute()}.
	\item La ruta optimizada es devuelta como \texttt{OptimizedRoute} y se envía al \textit{MapBloc}, que se encarga de gestionar la actualización del mapa en la interfaz de usuario.
	\item Finalmente, el \textit{MapBloc} utiliza la ruta optimizada para invocar el método \texttt{updateMap()}, mostrando el tour generado en \textit{MapScreen} y permitiendo al usuario interactuar con los puntos de interés.
\end{enumerate}
\imagen{secuence}{Diagrama de secuencia - Generación de Eco City Tour}

\subsection{Añadir un \acrfull{pdi} (CU02.3)}
El flujo para añadir un nuevo \acrshort{pdi} corresponde al \textbf{CU02.3 Añadir PDI}. El proceso se inicia cuando el usuario selecciona la opción de añadir un \acrshort{pdi} desde la interfaz de usuario \textit{MapScreen}. 

Este evento desencadena una serie de interacciones entre los componentes principales:
\begin{itemize}
	\item El \textit{MapBloc} gestiona el estado del mapa.
	\item El \textit{TourBloc} actualiza la lista de \acrshort{pdi} en el tour actual.
\end{itemize}
Finalmente, el sistema actualiza la vista del mapa con el \acrshort{pdi} añadido y guarda los cambios en el repositorio correspondiente.

La Figura~\ref{fig:add-poi-secuence} muestra el diagrama de secuencia de esta funcionalidad.

\imagen{add-poi-secuence}{Diagrama de secuencia - Añadir un \acrshort{pdi}}

\subsection{Eliminar un \acrshort{pdi} (CU02.2)}
El flujo para eliminar un \acrshort{pdi}, relacionado con el \textbf{CU02.2 Eliminar PDI}, permite al usuario ajustar su ruta eliminando aquellos \acrshort{pdi} que no sean relevantes.

El usuario inicia este flujo seleccionando el \acrshort{pdi} a eliminar desde:
\begin{itemize}
	\item \textit{MapScreen} a través del \textit{custom\_bottom\_sheet}.
	\item La pantalla de resumen (\textit{poi\_list\_item}).
\end{itemize}
Ambos eventos lanzan una actualización en el \textit{TourBloc}, que solicita al \textit{Google Directions Service} recalcular la ruta optimizada. Finalmente, el \textit{MapBloc} actualiza la vista del mapa.

La Figura~\ref{fig:remove-poi-secuence} muestra el diagrama de secuencia correspondiente.

\imagen{remove-poi-secuence}{Diagrama de secuencia - Eliminar un \acrshort{pdi}}

\subsection{Guardar un Eco City Tour (CU04)}
La funcionalidad de guardar un Eco City Tour, asociada al \textbf{CU04 Guardar Eco City Tour}, asegura que los usuarios puedan acceder a sus rutas en el futuro.

El flujo comienza cuando el usuario selecciona la opción de guardar desde la interfaz \textit{SummaryScreen}. Este evento activa el \textit{TourBloc}, que envía los datos al repositorio, donde se almacenan a través del \textit{FirestoreDataset}. Una vez completado el guardado, se notifica al usuario mediante un \textit{Snackbar}.

La Figura~\ref{fig:save-tour-secuence} muestra el diagrama de secuencia correspondiente.

\imagen{save-tour-secuence}{Diagrama de secuencia - Guardar un Eco City Tour}

\subsection{Cargar un Eco City Tour (CU05)}
La funcionalidad de cargar un tour previamente guardado está relacionada con el \textbf{CU05 Cargar Eco City Tour}.

El flujo inicia cuando el usuario selecciona un tour desde la lista de tours guardados en la interfaz \textit{TourSelectionScreen}. Este evento activa el \textit{TourBloc}, que recupera los datos del tour desde el repositorio (\textit{FirestoreDataset}). Una vez recuperada la información, el \textit{MapBloc} actualiza la vista del mapa para mostrar el tour cargado.

La Figura~\ref{fig:load-tour-secuence} describe el diagrama de secuencia para este proceso.

\imagen{load-tour-secuence}{Diagrama de secuencia - Cargar un Eco City Tour}



\subsection{Diseño arquitectónico}

El diseño de datos de la aplicación \textit{Eco City Tours} se fundamenta en una arquitectura modular que organiza las diferentes responsabilidades del sistema en paquetes específicos. A continuación, se presenta un diagrama de clases en la Figura~\ref{fig:clases} que detalla la estructura del sistema y sus relaciones. Este diagrama ilustra cómo interactúa la aplicación usando servicios, modelos y repositorios.

La organización en paquetes garantiza un diseño cohesivo, con bajo acoplamiento y alta cohesión, la extensibilidad del sistema coincidiendo también con cada carpeta dentro de la estructura del proyecto Flutter lo que garantiza también una fácil mantenibilidad.

\imagen{clases}{Diagrama de clases de la aplicación}

A continuación, se describen los paquetes y componentes principales representados en el diagrama de clases:

\begin{itemize}
	\item \textbf{Services}:
	\begin{itemize}
		\item \textbf{GeminiService}: Servicio encargado de interactuar con el modelo LLM \textit{Gemini} para obtener información sobre \acrlong{pdi} y generar recomendaciones personalizadas.
		\item \textbf{PlacesService}: Proporciona datos relacionados con lugares de interés mediante la integración con la API de \textit{Google Places}.
		\item \textbf{OptimizationService}: Se encarga de calcular rutas optimizadas entre puntos de interés, teniendo en cuenta las preferencias del usuario y criterios sostenibles.
	\end{itemize}
	
	\item \textbf{google maps flutter}:
	\begin{itemize}
		\item Clase que administra la interacción con los mapas en la interfaz de usuario, permitiendo mostrar las rutas generadas.
	\end{itemize}
	
	\item \textbf{models}: existen dos clases principales en las que se basa el diseño de la aplicación.
	\begin{itemize}
		\item \textbf{\acrlong{pdi}}: Clase que modela un \acrshort{pdi}, almacenando información relevante como nombre, ubicación y descripción. 
		\item \textbf{EcoCityTour}: Clase que representa un tour turístico completo, que incluye una lista del modelo anterior y datos como duración, distancia y nombre del lugar donde se ha generado la ruta turística.
	\end{itemize}
	
	\item \textbf{blocs}: este paquete gestionará la lógica del gestor de estados de la aplicación.
	\begin{itemize}
		\item \textbf{TourBloc}: Responsable de la gestión del estado relacionado con los tours, incluyendo la generación y modificación de rutas.
		\item \textbf{MapBloc}: Administra el estado relacionado con la visualización en el mapa, como el trazado de rutas.
	\end{itemize}
	
	\item \textbf{repositories}:
	\begin{itemize}
		\item \textbf{EcoCityTourRepository}: Implementa la lógica necesaria para, guardar y cargar información de un \textit{EcoCityTour}.
	\end{itemize}
	
	\item \textbf{datasets}:
	\begin{itemize}
		\item \textbf{FirestoreDataset}: Clase que gestiona la persistencia de datos en la base de datos en este caso concreto de Firestore.
	\end{itemize}
\end{itemize}

Con esta organización, el diseño asegura que cada componente tenga una responsabilidad clara, permitiendo la integración fluida de servicios externos, la manipulación eficiente de datos y la presentación interactiva de la información en la interfaz de usuario. 

Cabe destacar que al ser modular cualquier modificación por ejemplo de un gestor de estado o un dataset diferente se facilita enormemente.


El diseño arquitectónico de \textit{Eco City Tours} sigue un enfoque modular que organiza el sistema en tres capas principales: \textbf{Interfaz de usuario}, \textbf{Lógica de negocio} y \textbf{Servicios externos}. Esta separación garantiza la escalabilidad, mantenibilidad y claridad del sistema, facilitando futuras mejoras o adaptaciones.

\begin{itemize}
	\item \textbf{Interfaz de usuario}: Representa la capa con la que interactúa el usuario. Incluye componentes como \textit{MapScreen}, \textit{TourSummary} y \textit{TourSelectionScreen}, que se encargan de capturar las acciones del usuario y mostrar la información relevante en una interfaz intuitiva.
	
	\item \textbf{Lógica de negocio}: Se encarga de gestionar los procesos centrales de la aplicación. Los principales bloques funcionales son:
	\begin{itemize}
		\item \textit{MapBloc}: Gestiona el estado y las interacciones relacionadas con el mapa, como la visualización de las rutas o la actualización del mapa en tiempo real.
		\item \textit{TourBloc}: Responsable de la generación y modificación de tours, así como de la interacción con los datos necesarios para personalizar las rutas.
	\end{itemize}
	
	\item \textbf{Servicios externos}: Contiene los módulos que interactúan con servicios de terceros para obtener información necesaria para el funcionamiento de la aplicación. Los servicios externos principales son:
	\begin{itemize}
		\item \textit{Servicio LLM}: Se utiliza para generar datos iniciales de puntos de interés turísticos basados en las preferencias del usuario.
		\item \textit{Google Places}: Proporciona información enriquecida sobre los puntos de interés.
		\item \textit{Optimizador de Rutas}: Calcula la mejor ruta posible basándose en criterios como sostenibilidad y tiempo.
	\end{itemize}
\end{itemize}

Esta arquitectura asegura un sistema robusto y flexible, con componentes que tienen responsabilidades claramente definidas y un bajo acoplamiento entre ellos lo que permitiría un remplazo de uno de los componentes sin gran dificultad.

\imagen{components-diagram}{Diagrama de componentes}

\subsection{Patrón de diseño \acrfull{bloc}}
El patrón de diseño \acrshort{bloc}~\cite{flutter_bloc} es fundamental en la arquitectura de la aplicación \textit{Eco City Tours}. Este patrón separa la lógica de negocio de la interfaz de usuario, facilitando el mantenimiento, la escalabilidad.

En el contexto de Flutter, el patrón \acrlong{bloc} utiliza un enfoque basado en flujos o \textit{streams} y eventos (ver Fig.~\ref{fig:bloc-diagram}), permitiendo que los estados y eventos fluyan entre los componentes sin acoplar directamente la lógica de negocio con la interfaz gráfica. 

La aplicación \textit{Eco City Tours} implementa este patrón de diseño a través de los componentes principales, cada uno con una funcionalidad clara y única.
\begin{itemize}
	\item \textbf{\textit{GpsBloc}}: Gestiona tanto los permisos de uso de GPS como su activación actual en el dispositivo donde esté funcionando la aplicación.
	\item \textbf{\textit{LocationBloc}}: Gestiona la localización del usuario a través de un stream que guarda la posición GPS por la que va circulando.
	\item \textbf{\textit{TourBloc}}: Gestiona toda la lógica relacionada con los tours, desde la generación y modificación de rutas hasta la interacción con los servicios externos como \textit{Servicio LLM} y \textit{Google Places}.
	\item \textbf{\textit{MapBloc}}: Gestiona el estado y las interacciones del mapa, como la visualización de rutas generadas, la actualización de puntos de interés.
\end{itemize}

Un aspecto importante en la implementación del patrón \acrshort{bloc} en \textit{Eco City Tours} es el uso de la clase \texttt{Equatable}, que permite simplificar la comparación de estados y eventos. En Flutter, cada vez que un estado cambia, se debe notificar a la interfaz de usuario para que se actualice. La clase \texttt{Equatable} facilita este proceso al definir de manera eficiente si dos instancias de estado o evento son equivalentes, evitando notificaciones redundantes y mejorando el rendimiento del sistema.

En el caso de \textit{Eco City Tours}, todos los \acrshort{bloc} usados extienden de \texttt{Equatable}, asegurando que la lógica de comparación sea precisa y que la interfaz de usuario solo se actualice cuando sea necesario. 

\subsubsection{Principios clave del patrón \acrshort{bloc}}
	El patrón \acrfull{bloc} sigue los siguientes principios:
	\begin{itemize}
		\item \textbf{Eventos y estados}: La interacción entre la interfaz de usuario y la lógica de negocio se realiza mediante el envío de eventos (\texttt{Event}) y la emisión de estados (\texttt{State}). Los eventos representan acciones realizadas por el usuario, mientras que los estados reflejan el resultado de procesar esos eventos.
		\item \textbf{Flujos (\textit{Streams})}: El patrón utiliza \textit{streams} de datos reactivos para manejar la comunicación entre la interfaz de usuario y la lógica de negocio.
		\item \textbf{Inmutabilidad}: Tanto los eventos como los estados son inmutables, lo que asegura un comportamiento predecible y simplifica el manejo del flujo de datos.
	\end{itemize}
	
	\imagen{bloc-diagram}{Diagrama de interacción del patrón \acrshort{bloc} en la aplicación}

\subsubsection{Ejemplo de implementación en \textit{Eco City Tours}}
A continuación, se muestra un ejemplo simplificado de cómo se utiliza el patrón \acrshort{bloc} para gestionar la generación de un tour:
\begin{itemize}
	\item El usuario selecciona sus preferencias y dispara un evento \texttt{loadTourEvent()} desde la interfaz de usuario.
	\item El \textit{TourBloc} recibe este evento, interactúa con los servicios externos (\textit{Servicio LLM} y \textit{Google Places}) para obtener y enriquecer los puntos de interés, y calcula una ruta optimizada.
	\item El \textit{TourBloc} emite un nuevo estado (\texttt{TourLoadedState}) con la información del tour generado, que es recibido por el componente \textit{MapBloc}.
	\item El \textit{MapBloc} utiliza esta información para actualizar el mapa, invocando el método \texttt{updateMap()}.
\end{itemize}
Como se puede ver la lógica del tour generado y su representación son dos partes diferenciadas aunque MapBloc dependa del contenido del TourBloc, su codificación no se recarga delegando responsabilidades.