\apendice{Especificación de diseño}

\section{Introducción}
En este anexo, se detallan las especificaciones de diseño del proyecto, enfocadas en los aspectos fundamentales para el desarrollo de la aplicación. Se describen cómo se organizan los datos que lo componen, el diseño arquitectónico, y los procedimientos empleados. Estas especificaciones son clave para asegurar el correcto funcionamiento y la estructuración adecuada de cada uno de los elementos que componen \textit{Eco City Tours}.
\section{Diseño de datos}

El diseño de datos de la aplicación \textit{Eco City Tours} se fundamenta en una arquitectura modular que organiza las diferentes responsabilidades del sistema en paquetes específicos. A continuación, se presenta un diagrama de clases en la Figura~\ref{fig:clases} que detalla la estructura del sistema y sus relaciones. Este diagrama ilustra cómo interactúa la aplicación usando de servicios, modelos y repositorios.

La organización en paquetes garantiza un diseño cohesivo, con bajo acoplamiento y alta cohesión, la extensibilidad del sistema coincidiendo también con cada carpeta dentro de la estructura del proyecto Flutter lo que garantiza también una fácil mantenibilidad.

\imagen{clases}{Diagrama de clases de la aplicación}

A continuación, se describen los paquetes y componentes principales representados en el diagrama de clases:

\begin{itemize}
	\item \textbf{Services}:
	\begin{itemize}
		\item \textbf{GeminiService}: Servicio encargado de interactuar con el modelo LLM \textit{Gemini} para obtener información sobre \acrlong{pdi} y generar recomendaciones personalizadas.
		\item \textbf{PlacesService}: Proporciona datos relacionados con lugares de interés mediante la integración con la API de \textit{Google Places}.
		\item \textbf{OptimizationService}: Se encarga de calcular rutas optimizadas entre puntos de interés, teniendo en cuenta las preferencias del usuario y criterios sostenibles.
	\end{itemize}
	
	\item \textbf{google maps flutter}:
	\begin{itemize}
		\item Clase que administra la interacción con los mapas en la interfaz de usuario, permitiendo mostrar las rutas generadas.
	\end{itemize}
	
	\item \textbf{models}: existen dos clases principales en las que se basa el diseño de la aplicación.
	\begin{itemize}
		\item \textbf{\acrshort{pdi}ntOfInterest}: Clase que modela un \acrlong{pdi}, almacenando información relevante como nombre, ubicación y descripción. 
		\item \textbf{EcoCityTour}: Clase que representa un tour turístico completo, que incluye una lista del modelo anterior y datos como duración, distancia y nombre del lugar donde se ha generado la ruta turística.
	\end{itemize}
	
	\item \textbf{blocs}: este paquete gestionará la lógica del gestor de estados de la aplicación.
	\begin{itemize}
		\item \textbf{TourBloc}: Responsable de la gestión del estado relacionado con los tours, incluyendo la generación y modificación de rutas.
		\item \textbf{MapBloc}: Administra el estado relacionado con la visualización en el mapa, como el trazado de rutas.
	\end{itemize}
	
	\item \textbf{repositories}:
	\begin{itemize}
		\item \textbf{EcoCityTourRepository}: Implementa la lógica necesaria para, guardar y cargar información de un \textit{EcoCityTour}.
	\end{itemize}
	
	\item \textbf{datasets}:
	\begin{itemize}
		\item \textbf{FirestoreDataset}: Clase que gestiona la persistencia de datos en la base de datos en este caso concreto de Firestore.
	\end{itemize}
\end{itemize}

Con esta organización, el diseño asegura que cada componente tenga una responsabilidad clara, permitiendo la integración fluida de servicios externos, la manipulación eficiente de datos y la presentación interactiva de la información en la interfaz de usuario. 

Cabe destacar que al ser modular cualquier modificación por ejemplo de un gestor de estado o un dataset diferente se facilita enormemente.



\section{Diseño procedimental}
En esta sección se explicará el flujo de trabajo relacionado con una de las tareas más importantes de la aplicación.
\subsection{Carga de \textit{EcoCityTour} desde pantalla de selección de preferencias}
La generación de un nuevo tour turístico una vez que el usuario ha definido sus preferencias es el proceso fundamental que mostraremos en la siguiente Figura~\ref{fig:secuence} que mostrará la interacción entre los distintos actores y sus comunicaciones.

El proceso comienza con el evento \texttt{loadTourEvent()}, disparado por el usuario desde la pantalla de selección (\textit{TourSelectionScreen}). Este evento desencadena la siguiente secuencia de interacciones:
\begin{enumerate}
	\item El \textit{TourBloc} inicia la generación del tour solicitando a \textit{ServicioLLM} una lista inicial de \acrlong{pdi} basados en las preferencias del usuario.
	\item \textit{ServicioLLM} devuelve un conjunto de \acrlong{pdi}, que luego se enriquecen con información adicional obtenida a través de \textit{Google Places} mediante el método \texttt{searchPlacePOI}.
	\item La información enriquecida de los \acrlong{pdi} es enviada a \textit{OptimizationService}, que calcula una ruta optimizada utilizando criterios de sostenibilidad y las preferencias del usuario. Esto se realiza a través del método \texttt{optimizedRoute()}.
	\item La ruta optimizada es devuelta como \texttt{OptimizedRoute} y se envía al \textit{MapBloc}, que se encarga de gestionar la actualización del mapa en la interfaz de usuario.
	\item Finalmente, el \textit{MapBloc} utiliza la ruta optimizada para invocar el método \texttt{updateMap()}, mostrando el tour generado en \textit{MapScreen} y permitiendo al usuario interactuar con los puntos de interés.
\end{enumerate}
\imagen{secuence}{Diagrama de secuencia - Generación de Eco City Tour}
Además de este proceso principal de obtención de \acrlong{pdi} existen otros casos de uso cuyos procedimientos se explican a continuación.
\subsection{Añadir un \acrfull{pdi}}

El flujo comienza cuando el usuario selecciona la opción de añadir un \acrshort{pdi} desde la interfaz de usuario \textit{MapScreen}. Este evento desencadena una serie de interacciones entre los componentes principales, como el \textit{MapBloc}, que gestiona el estado del mapa, y el \textit{TourBloc}, que actualiza la lista de \acrshort{pdi} en el tour actual. Finalmente, el sistema actualiza la vista del mapa con el \acrshort{pdi} añadido y guarda los cambios en el repositorio correspondiente.

En el diagrama de secuencia representado en la Figura~\ref{fig:add-poi-secuence}, se describen los pasos necesarios para completar esta funcionalidad.

\imagen{add-poi-secuence}{Diagrama de secuencia - Añadir un \acrshort{pdi}}

La lógica de la pantalla de búsqueda se beneficia de una clase de \textit{Flutter} llamada \textit{Delegate}~\cite{flutter_search_delegate}. En esta clase se pueden configurar los comportamientos típicos de una barra de búsqueda fácilmente sin tener que programar cada interacción manualmente como por ejemplo la impresión de resultados, las sugerencias de lugares o los iconos mostrados.

\subsection{Eliminar un \acrshort{pdi}}
El flujo para eliminar un \acrshort{pdi} permite al usuario ajustar su ruta turística eliminando aquellos \acrshort{pdi} que no le resulten  relevantes. 

El usuario inicia este flujo seleccionando el \acrshort{pdi} a eliminar desde la interfaz de usuario \textit{MapScreen} a través del custom\_bottom\_sheet o desde la pantalla de resumen en poi\_list\_item, ambos van a lanzar un evento al TourBloc que actualizará su información usando el actor 5 que es el servicio de Google que optimizará la ruta con la actualización del tour. Por último delegará como siempre la actualización visual del mapa al \textit{MapBloc}. 

La Figura~\ref{fig:remove-poi-secuence} muestra el diagrama de secuencia para esta funcionalidad.

\imagen{remove-poi-secuence}{Diagrama de secuencia - Eliminar un \acrshort{pdi}}

\subsection{Guardar un Eco City Tour}
Esta funcionalidad permite a los usuarios guardar un Eco City Tour generado, asegurando que puedan acceder a sus rutas personalizadas en el futuro. 

El flujo comienza como se puede observar en la Figura~\ref{fig:save-tour-secuence} cuando el usuario selecciona la opción de guardar desde la interfaz de usuario en la pantalla \textit{SummaryScreen}. Este evento activa el \textit{TourBloc}. Los datos son entonces enviados al repositorio, donde son almacenados de manera persistente a través de las funciones del Datasource de Firestore y por último este actor realizará la acción persistente. Una vez completado el proceso, se notifica al usuario que el tour ha sido guardado exitosamente a través de un \textit{Snackbar}, un mensaje en la parte inferior de la pantalla que servirá para su confirmación.
\imagen{save-tour-secuence}{Diagrama de secuencia - Guardar un Eco City Tour}

\subsection{Cargar un Eco City Tour}
Esta funcionalidad permite a los usuarios cargar un tour previamente guardado para continuar utilizándolo o editarlo.

El flujo inicia cuando el usuario selecciona un tour desde la lista de tours guardados en la interfaz de usuario (\textit{TourSelectionScreen}). Este evento activa el \textit{TourBloc}, que recupera los datos del tour desde el repositorio correspondiente. Una vez recuperados los datos, el \textit{MapBloc} actualiza la vista del mapa para mostrar el tour cargado. 

La Figura~\ref{fig:load-tour-secuence} muestra el diagrama de secuencia para este proceso.

\imagen{load-tour-secuence}{Diagrama de secuencia - Cargar un Eco City Tour}


\subsection{Diseño arquitectónico}

El diseño arquitectónico de \textit{Eco City Tours} sigue un enfoque modular que organiza el sistema en tres capas principales: \textbf{Interfaz de usuario}, \textbf{Lógica de negocio} y \textbf{Servicios externos}. Esta separación garantiza la escalabilidad, mantenibilidad y claridad del sistema, facilitando futuras mejoras o adaptaciones.

\begin{itemize}
	\item \textbf{Interfaz de usuario}: Representa la capa con la que interactúa el usuario. Incluye componentes como \textit{MapScreen}, \textit{TourSummary} y \textit{TourSelectionScreen}, que se encargan de capturar las acciones del usuario y mostrar la información relevante en una interfaz intuitiva.
	
	\item \textbf{Lógica de negocio}: Se encarga de gestionar los procesos centrales de la aplicación. Los principales bloques funcionales son:
	\begin{itemize}
		\item \textit{MapBloc}: Gestiona el estado y las interacciones relacionadas con el mapa, como la visualización de las rutas o la actualización del mapa en tiempo real.
		\item \textit{TourBloc}: Responsable de la generación y modificación de tours, así como de la interacción con los datos necesarios para personalizar las rutas.
	\end{itemize}
	
	\item \textbf{Servicios externos}: Contiene los módulos que interactúan con servicios de terceros para obtener información necesaria para el funcionamiento de la aplicación. Los servicios externos principales son:
	\begin{itemize}
		\item \textit{Servicio LLM}: Se utiliza para generar datos iniciales de puntos de interés turísticos basados en las preferencias del usuario.
		\item \textit{Google Places}: Proporciona información enriquecida sobre los puntos de interés.
		\item \textit{Optimizador de Rutas}: Calcula la mejor ruta posible basándose en criterios como sostenibilidad y tiempo.
	\end{itemize}
\end{itemize}

Esta arquitectura asegura un sistema robusto y flexible, con componentes que tienen responsabilidades claramente definidas y un bajo acoplamiento entre ellos lo que permitiría un remplazo de uno de los componentes sin gran dificultad.

\imagen{components-diagram}{Diagrama de componentes}

\subsection{Patrón de diseño \acrfull{bloc}}
El patrón de diseño \acrshort{bloc}~\cite{flutter_bloc} es fundamental en la arquitectura de la aplicación \textit{Eco City Tours}. Este patrón separa la lógica de negocio de la interfaz de usuario, facilitando el mantenimiento, la escalabilidad.

En el contexto de Flutter, el patrón \acrlong{bloc} utiliza un enfoque basado en flujos o \textit{streams} y eventos (ver Fig.~\ref{fig:bloc-diagram}), permitiendo que los estados y eventos fluyan entre los componentes sin acoplar directamente la lógica de negocio con la interfaz gráfica. 

La aplicación \textit{Eco City Tours} implementa este patrón de diseño a través de los componentes principales, cada uno con una funcionalidad clara y única.
\begin{itemize}
	\item \textbf{\textit{GpsBloc}}: Gestiona tanto los permisos de uso de GPS como su activación actual en el dispositivo donde esté funcionando la aplicación.
	\item \textbf{\textit{LocationBloc}}: Gestiona la localización del usuario a través de un stream que guarda la posición GPS por la que va circulando.
	\item \textbf{\textit{TourBloc}}: Gestiona toda la lógica relacionada con los tours, desde la generación y modificación de rutas hasta la interacción con los servicios externos como \textit{Servicio LLM} y \textit{Google Places}.
	\item \textbf{\textit{MapBloc}}: Gestiona el estado y las interacciones del mapa, como la visualización de rutas generadas, la actualización de puntos de interés.
\end{itemize}

Un aspecto importante en la implementación del patrón \acrshort{bloc} en \textit{Eco City Tours} es el uso de la clase \texttt{Equatable}, que permite simplificar la comparación de estados y eventos. En Flutter, cada vez que un estado cambia, se debe notificar a la interfaz de usuario para que se actualice. La clase \texttt{Equatable} facilita este proceso al definir de manera eficiente si dos instancias de estado o evento son equivalentes, evitando notificaciones redundantes y mejorando el rendimiento del sistema.

En el caso de \textit{Eco City Tours}, todos los \acrshort{bloc} usados extienden de \texttt{Equatable}, asegurando que la lógica de comparación sea precisa y que la interfaz de usuario solo se actualice cuando sea necesario. 

\subsubsection{Principios clave del patrón \acrshort{bloc}}
	El patrón \acrfull{bloc} sigue los siguientes principios:
	\begin{itemize}
		\item \textbf{Eventos y estados}: La interacción entre la interfaz de usuario y la lógica de negocio se realiza mediante el envío de eventos (\texttt{Event}) y la emisión de estados (\texttt{State}). Los eventos representan acciones realizadas por el usuario, mientras que los estados reflejan el resultado de procesar esos eventos.
		\item \textbf{Flujos (\textit{Streams})}: El patrón utiliza \textit{streams} de datos reactivos para manejar la comunicación entre la interfaz de usuario y la lógica de negocio.
		\item \textbf{Inmutabilidad}: Tanto los eventos como los estados son inmutables, lo que asegura un comportamiento predecible y simplifica el manejo del flujo de datos.
	\end{itemize}
	
	\imagen{bloc-diagram}{Diagrama de interacción del patrón \acrshort{bloc} en la aplicación}

\subsubsection{Ejemplo de implementación en \textit{Eco City Tours}}
A continuación, se muestra un ejemplo simplificado de cómo se utiliza el patrón \acrshort{bloc} para gestionar la generación de un tour:
\begin{itemize}
	\item El usuario selecciona sus preferencias y dispara un evento \texttt{loadTourEvent()} desde la interfaz de usuario.
	\item El \textit{TourBloc} recibe este evento, interactúa con los servicios externos (\textit{Servicio LLM} y \textit{Google Places}) para obtener y enriquecer los puntos de interés, y calcula una ruta optimizada.
	\item El \textit{TourBloc} emite un nuevo estado (\texttt{TourLoadedState}) con la información del tour generado, que es recibido por el componente \textit{MapBloc}.
	\item El \textit{MapBloc} utiliza esta información para actualizar el mapa, invocando el método \texttt{updateMap()}.
\end{itemize}
Como se puede ver la lógica del tour generado y su representación son dos partes diferenciadas aunque MapBloc dependa del contenido del TourBloc, su codificación no se recarga delegando responsabilidades.