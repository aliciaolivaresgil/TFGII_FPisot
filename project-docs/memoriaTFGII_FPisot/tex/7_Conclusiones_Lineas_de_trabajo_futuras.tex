\capitulo{7}{Conclusiones y Líneas de trabajo futuras}
\section*{Conclusiones}
\section*{Líneas de trabajo futuras}
Existen múltiples maneras de expandir y llenar de nuevas funcionalidades a la aplicación llevada a cabo.
Por citar algunas que puedan resultar más útiles al usuario:
\begin{itemize}
    \item \textbf{Gamificación:} Recompensas por rutas completadas o distancia recorrida con un medio ecológico. Logros que desbloqueen aspectos visuales del icono de la aplicación.
    \item \textbf{Ratings:} Valorar las tours generados permitiendo la búsqueda de los mismos a otros usuarios.
    \item \textbf{Mejora en planificador de rutas:} Teniendo en cuenta otras fuentes de datos que mejoren la sostenibilidad de los tours como por ejemplo información satelital que determine las rutas en función de la zona de sombra.
    \item \textbf{Multiplataforma:} La aplicación podría beneficiarse de su adaptación a otras plataformas, donde se tendría que tener en cuenta principalmente los permisos de localización. Al utilizar Flutter esta adaptación se podría realizar sobre el mismo código base, facilitando en gran medida su consecución.
	\item \textbf{Reconocimiento de lugares (Landmark Recognition):} Firebase ofrece una función avanzada de reconocimiento de lugares \cite{firebase_mlkit_landmarks} que permitiría añadir una capa de personalización en la aplicación. Esta funcionalidad permitiría al usuario tomar una foto de uno de los \acrlong{pdi} visitados, y la aplicación podría identificar automáticamente el lugar y actualizar el estado de la ruta en tiempo real, indicando que se ha completado la visita. Esta mejora no solo facilitaría el seguimiento de la ruta, sino que también proporcionaría una experiencia de usuario más interactiva y dinámica.
	\item \textbf{Registro y autenticación de usuarios:} Para implementar algunas de las mejoras propuestas, es necesaria una gestión integral de usuarios. Actualmente, no existe un proceso de registro o autenticación, ya que es Cloud Firestore quien asigna un identificador único a cada usuario. Aunque esta gestión permite identificar los Eco City Tours generados por cada usuario, presenta limitaciones, como la pérdida de acceso a los tours al reinstalar la aplicación, debido a la asignación de un nuevo identificador. Además, los tours previos quedarían en la base de datos sin posibilidad de ser recuperados. Implementar un sistema de registro y autenticación una vez configurado Firebase sería una tarea relativamente sencilla, que incluiría generar una pantalla de registro y acceso, así como modificar la configuración de Cloud Firestore para incluir módulos de autenticación mediante servicios como correo electrónico o cuentas de Google.
\end{itemize}