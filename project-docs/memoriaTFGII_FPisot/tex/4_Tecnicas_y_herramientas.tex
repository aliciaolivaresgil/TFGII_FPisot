\capitulo{4}{Técnicas y herramientas}

Esta parte de la memoria tiene como objetivo presentar las técnicas metodológicas y las herramientas de desarrollo que se han utilizado para llevar a cabo el proyecto. Si se han estudiado diferentes alternativas de metodologías, herramientas, bibliotecas se puede hacer un resumen de los aspectos más destacados de cada alternativa, incluyendo comparativas entre las distintas opciones y una justificación de las elecciones realizadas. 
No se pretende que este apartado se convierta en un capítulo de un libro dedicado a cada una de las alternativas, sino comentar los aspectos más destacados de cada opción, con un repaso somero a los fundamentos esenciales y referencias bibliográficas para que el lector pueda ampliar su conocimiento sobre el tema.


\section{LangChain}
En el último año LangChain se ha establecido como uno de los marcos de trabajo más populares del mercado. Esta herramienta multiusos aúna aplicaciones tan necesarias para el mundo de los \acrfull{llm} como pueden ser base de datos de vectores, memoria, prompts, herramientas, agentes como ya hemos visto en la sección \ref{sec:agentes} y cadenas de pensamiento (de así su nombre chain). En el prototipo de prompting de este \acrshort{tfg} se puede ver el anidamiento de componentes como son estas cadenas para obtener la mejor entrada posible al modelo y obtener la mejor salida posible, estas cadenas pueden unir componentes como prompts, retrievers, processors, tools o incluso otras cadenas para procesos más complejos.
Con todo ello LangChain supone una manera de combinar el poder de los \acrshort{llm} con la lógica de cualquier aplicación.