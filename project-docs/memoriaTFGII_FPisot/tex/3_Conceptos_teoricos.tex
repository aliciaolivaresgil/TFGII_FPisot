\capitulo{3}{Conceptos teóricos}

En este capítulo se llevará a cabo una descripción de los conceptos necesarios para comprender el funcionamiento de la aplicación desarrollada. Es necesario tener un conocimiento acerca de los modelos del lenguaje que proporcionan el origen de datos en los que se basan los \acrfull{poi}.



\section{Large Language Models (LLM)}

Las secciones se incluyen con el comando section.

\subsection{Subsecciones}

Además de secciones tenemos subsecciones.

\subsubsection{Uso de LLMs en la aplicación de este TFG}

Y subsecciones. 


\section{Retrieval-Augmented Generation (RAG)}

\subsubsection{Uso de RAGs en la aplicación de este TFG}

Las referencias se incluyen en el texto usando cite~\cite{wiki:latex}. Para citar webs, artículos o libros~\cite{koza92}, si se desean citar más de uno en el mismo lugar~\cite{bortolot2005, koza92}.


\section{Imágenes}

Se pueden incluir imágenes con los comandos standard de \LaTeX, pero esta plantilla dispone de comandos propios como por ejemplo el siguiente:

\imagen{escudoInfor}{Autómata para una expresión vacía}{.5}



\section{Listas de items}

Existen tres posibilidades:

\begin{itemize}
	\item primer item.
	\item segundo item.
\end{itemize}

\begin{enumerate}
	\item primer item.
	\item segundo item.
\end{enumerate}

\begin{description}
	\item[Primer item] más información sobre el primer item.
	\item[Segundo item] más información sobre el segundo item.
\end{description}
	
\begin{itemize}
\item 
\end{itemize}

\section{Tablas}

Igualmente se pueden usar los comandos específicos de \LaTeX o bien usar alguno de los comandos de la plantilla.

\tablaSmall{Herramientas y tecnologías utilizadas en cada parte del proyecto}{l c c c c}{herramientasportipodeuso}
{ \multicolumn{1}{l}{Herramientas} & App AngularJS & API REST & BD & Memoria \\}{ 
HTML5 & X & & &\\
CSS3 & X & & &\\
BOOTSTRAP & X & & &\\
JavaScript & X & & &\\
AngularJS & X & & &\\
Bower & X & & &\\
PHP & & X & &\\
Karma + Jasmine & X & & &\\
Slim framework & & X & &\\
Idiorm & & X & &\\
Composer & & X & &\\
JSON & X & X & &\\
PhpStorm & X & X & &\\
MySQL & & & X &\\
PhpMyAdmin & & & X &\\
Git + BitBucket & X & X & X & X\\
Mik\TeX{} & & & & X\\
\TeX{}Maker & & & & X\\
Astah & & & & X\\
Balsamiq Mockups & X & & &\\
VersionOne & X & X & X & X\\
} 
Algunos conceptos teóricos de \LaTeX{} \footnote{Créditos a los proyectos de Álvaro López Cantero: Configurador de Presupuestos y Roberto Izquierdo Amo: PLQuiz}.