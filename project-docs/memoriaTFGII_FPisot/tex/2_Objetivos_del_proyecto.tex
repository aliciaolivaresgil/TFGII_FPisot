\capitulo{2}{Objetivos del proyecto}

\section{Objetivos funcionales}

Estos objetivos se centran en las funcionalidades y características que debe tener la aplicación Eco City Tours para satisfacer las necesidades y expectativas de los usuarios. A continuación se detallan los objetivos funcionales del proyecto:

\begin{itemize}
    \item \textbf{Propuesta de rutas turísticas personalizadas}: La aplicación debe ser capaz de generar rutas turísticas personalizadas basadas en las preferencias del visitante utilizando \acrfull{llm}. Para llevarlo a cabo, el usuario facilitará al modelo sus preferencias, eligiendo entre otras opciones el medio de transporte elegido o el número de \acrfull{pdi} a visitar.
    \item \textbf{Obtener los \acrfull{pdi}}: A través de la interacción con el modelo, la aplicación le indicará que debe priorizar un \acrshort{pdi} sobre otro en función de criterios sostenibles como la deslocalización del turismo y preferencias de usuario como puedan ser duración de la visita o medio de transporte ecológico a elegir.
    \item \textbf{Visualización de rutas en mapa}: La aplicación debe mostrar las rutas sugeridas en un mapa utilizando herramientas \acrshort{sig}.
    \item \textbf{Optimización para ciclistas y peatones}: la aplicación usará un servicio de navegación de calidad que debe ser capaz de calcular rutas seguras para peatones y priorizar carriles bicis sobre carreteras compartidas con vehículos motorizados.
    \item \textbf{Gestión de rutas}: La aplicación permitirá a los usuarios crear, guardar y compartir rutas turísticas, facilitando una mayor personalización y aprovechamiento de la experiencia turística.
    

\end{itemize}

\section{Objetivos no funcionales}

Los objetivos no funcionales se refieren a los desafíos y metas que se deben abordar para desarrollar el software. Estos objetivos abarcan aspectos como la arquitectura del sistema, las tecnologías a utilizar y las metodologías de desarrollo. A continuación se detallan los objetivos no funcionales del proyecto:

\begin{itemize}
	\item \textbf{Integración de inteligencia artificial usando \acrfull{nlp}}: la aplicación podrá configurarse con distintos modelos de lenguaje sin afectar su mantenimiento. El usuario \todo{el usuario no tendrá tal poder} tendrá la opción de decidir si los modelos LLM se cargan localmente o son accedidos a través de servicios de terceros, permitiendo flexibilidad en la configuración. Con la tecnología de LangChain, los modelos se consultarán para implementar los objetivos funcionales de la **propuesta de rutas turísticas personalizadas** y **obtención de los Puntos de Interés (PDI)**. Esta arquitectura permite una integración robusta y adaptable, evaluando los resultados generados por los modelos para asegurar la precisión y relevancia en la creación de rutas.
    \item \textbf{Uso de Herramientas Open-Source}: se priorizará para el desarrollo de la aplicación programas, paquetes, servicios o librerías que sean de código abierto, siempre que sea posible y no repercuta en la calidad del producto final. Se priorizará en todo caso una solución que no incurra en gasto alguno para el desarrollador o al usuario final por su uso. De esta manera se intenta fomentar el \acrshort{ods} 4: \textit{Educación de calidad} que busca garantizar una educación inclusiva, equitativa y de calidad y promover oportunidades de aprendizaje para todos.
    \item \textbf{Usabilidad}: la interacción del usuario con la aplicación debe ser intuitiva y sencilla, permitiendo un rápido aprendizaje de todas sus funcionalidades. El diseño de la interfaz debe estar orientado a ofrecer una experiencia de uso fluida. Debe ser también, altamente configurable para parametrizar todas las preferencias de usuario adaptándose a sus gustos.
\end{itemize}

\section{Objetivos personales}


\begin{itemize}
	\item \textbf{Formación en \acrshort{llm} y su integración en aplicaciones software.} Dada la rápida evolución de los \acrfull{llm} y la amplitud de campos del conocimiento en los que se pueden utilizar, obtener una base de conocimientos destacable en este área sería un objetivo que me permitiría expandir mi futuro académico y por tanto distinguir mi perfil profesional especializándome en un sector con fuerte expansión.
	\item \textbf{Desarrollo de aplicación móvil profesional}: poner en práctica lo aprendido en varios cursos de auto-formación online en \textbf{Dart y Flutter.} La aplicación de este proyecto puede ser parte de mi porfolio con aplicaciones que muestren mis habilidades a futuros empleadores.
	\item \textbf{Finalización del \acrshort{tfg} y Grado}: tras no haber completado la Ingeniería Técnica Informática en su momento por no haber realizado el Proyecto Fin de Carrera, la realización de este \acrshort{tfg} marca la culminación de mi formación académica como ingeniero.
\end{itemize}
