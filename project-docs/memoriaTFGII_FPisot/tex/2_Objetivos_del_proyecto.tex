\capitulo{2}{Objetivos del proyecto}

\section{Objetivos Funcionales}

Estos objetivos se centran en las funcionalidades y características que debe tener la aplicación para satisfacer las necesidades y expectativas de los usuarios. A continuación se detallan los objetivos funcionales del proyecto:

\begin{itemize}
    \item \textbf{Propuesta de Rutas Turísticas Personalizadas}: La aplicación debe ser capaz de generar rutas turísticas personalizadas basadas en las preferencias del usuario utilizando modelos de lenguaje de gran escala (\acrfull{llm}) y el framework \textbf{LangChain}.
    \item \textbf{Obtener los Puntos de Interés (\acrfull{poi})}: La aplicación debe identificar y conectar diversos puntos de interés, proporcionando información relevante sobre cada uno.
    \item \textbf{Visualización de Rutas en Mapa}: La aplicación debe mostrar las rutas sugeridas en un mapa utilizando herramientas libres como \acrfull{osm}.
    \item \textbf{Optimización para Ciclistas y Peatones}: La aplicación debe promover la movilidad sostenible sugiriendo rutas optimizadas para ciclistas y peatones.
    \item \textbf{Interfaz Intuitiva y Amigable}: El usuario debe interactuar con la aplicación de manera intuitiva, siendo fácil de usar por los usuarios las diferentes funcionalidades.
\end{itemize}

\section{Objetivos Técnicos}

Los objetivos técnicos se refieren a los desafíos y metas técnicas que se deben abordar para desarrollar el software. Estos objetivos abarcan aspectos como la arquitectura del sistema, las tecnologías a utilizar y las metodologías de desarrollo. A continuación se detallan los objetivos técnicos del proyecto:

\begin{itemize}
    \item \textbf{Implementación de \acrfull{llm} y \textbf{LangChain}}: Integrar modelos de lenguaje a gran escala (\acrshort{llm}) y el framework \textbf{LangChain} para la generación de rutas y procesamiento de información relevante y ser capaz de integrar dicho conocimiento para ser mostrada en la aplicación móvil así como en un prototipo que muestre de manera incremental la mejora obtenida por parte de los modelos usando diferentes técnicas a la hora de interactuar con ellos como puede ser el uso de técnicas \acrfull{rag}, agentes, etc.
    \item \textbf{Uso de Herramientas Open-Source}: Emplear medios abiertos, libres y gratuitos como \acrfull{osm} para la visualización de mapas y rutas en vez de utilizar servicios que puedan incurrir en gastos para el usuario. 
\end{itemize}

\section{Objetivos Personales}

\begin{itemize}
	\item \textbf{Conocimiento avanzado en \acrshort{llm}}: dada la evolución de esta tecnología, los amplios campos en los que se puede utilizar, obtener una base de conocimientos sería un objetivo que me permitiría expandir mi futuro académico y por tanto distinguir mi perfil profesional especializándome en este sector que se encuentra en fuerte expansión.
	\item \textbf{Creación de aplicación móvil profesional}: de igual manera poner en práctica lo aprendido en varios cursos de \textbf{Dart y Flutter} puede contribuir a que la aplicación de este proyecto sea parte de mi porfolio con aplicaciones que muestren mis habilidades a futuros empleadores.
	\item \textbf{Consecución del \acrshort{tfg} y conclusión de Grado}: al no haber terminado la Ingeniería Técnica Informática por no haber realizado un Proyecto Fin de Carrera en mi pasado, la consecución de este \acrshort{tfg} sirve para convertirme en ingeniero al ser la última asignatura del Grado y supone la consecución de una carga personal de casi veinte años.
\end{itemize}
