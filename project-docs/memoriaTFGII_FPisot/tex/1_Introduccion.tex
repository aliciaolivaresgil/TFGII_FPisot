\capitulo{1}{Introducción}

El crecimiento de población en las ciudades \cite{nieuwenhuijsen_urban_2020} y el turismo como catalizador de la gentrificación supone el gran campo de batalla para gobiernos locales de los países occidentales que han visto como la falta de una legislación controlada del turismo supone un grave problema afectando múltiples niveles de la convivencia, economía y el medio ambiente. A pesar de los avances en la promoción de un nuevo modelo urbano, muchas urbes aún enfrentan desafíos significativos en la integración de prácticas sostenibles en la vida cotidiana de sus habitantes. La falta de información accesible y personalizada sobre rutas y actividades que promuevan la movilidad sostenible y el turismo responsable es un marco común que se debe desarrollar si se quiere evitar que el conflicto crezca sin fin. Esta brecha de información impide que tanto residentes como turistas adopten hábitos más sostenibles que beneficien a la comunidad local y al medio ambiente en un marco global.

Fomentar el Turismo Sostenible supone una gran oportunidad para intentar contrarrestar la deriva actual. Y es que el turismo es un motor fundamental de la economía a nivel global y por tanto tiene la capacidad de transformarse para ayudar a la sostenibilidad del planeta.  Destaca en este marco de trabajo el siguiente objetivo \acrshort{ods} identificado como número 11 que se centra en hacer que las ciudades y los asentamientos humanos sean sostenibles. Según el informe de la \textit{UNESCO}, \cite{ionescu_progress_2024}, no solo es crucial por sí mismo este \acrshort{ods}, sino que actúa como un multiplicador, influyendo indirectamente en la consecución de otros \acrshort{ods} debido a su enfoque integral y transversal.

En este contexto, la aplicación móvil desarrollada Eco City Tours \textbf{aúna estos esfuerzos al proporcionar una herramienta práctica y accesible} para la promoción del ODS11 y la movilidad sostenible. La aplicación Eco City Tours producto de este \acrshort{tfg}, desarrollada en Flutter, utiliza \acrfull{llm} para generar rutas turísticas personalizadas que conecten \acrfull{poi}. La aplicación se enfoca en las preferencias del usuario, ofreciendo rutas optimizadas para ciclistas y peatones promoviendo así la movilidad sostenible. Eco City Tours durante el proceso de obtención de los \acrfull{poi} requiere al \acrshort{llm} que utilice prácticas sostenibles como la deslocalización del turismo, esta solución no solo enriquece la experiencia turística \cite{mitas_tell_2023}, sino que también promueve prácticas sostenibles, que en lugar de perjudicar a los visitantes, puede apoyar el sustento de las comunidades locales. De este modo, se logra un impacto positivo tanto a nivel local como en el medio ambiente a escala global.