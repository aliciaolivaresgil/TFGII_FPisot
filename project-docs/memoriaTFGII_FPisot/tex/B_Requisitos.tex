\apendice{Especificación de Requisitos}

\section{Introducción}
En esta sección se presentan los requisitos de la aplicación, abordando tanto los objetivos generales como los específicos del proyecto. Se incluye un catálogo detallado de los requisitos funcionales y no funcionales, que definen el comportamiento y las características técnicas de la aplicación. Además, se proporciona una especificación detallada de los requisitos a través de tablas de casos de uso, complementadas con su respectivo diagrama de casos de uso, lo que facilita una comprensión clara de las interacciones principales de los usuarios con el sistema.


\section{Objetivos generales}
La misión fundamental de este proyecto persigue conseguir los siguientes propósitos:
\begin{itemize}
	\item \textbf{Fomentar el turismo sostenible:} Facilitar a los usuarios la exploración de ciudades promoviendo al mostrar rutas no motorizadas y modos de transporte como caminar y el uso de bicicletas.
	
	\item \textbf{Optimización de experiencias turísticas personalizadas:} Ofrecer a los usuarios rutas personalizadas que se ajusten a sus intereses y preferencias, proporcionando información detallada y relevante sobre los puntos de interés seleccionados.
	
	\item \textbf{Promover el uso de tecnologías inteligentes en el turismo:} Utilizar tecnologías avanzadas como servicios GIS, Google Places y LLMs para mejorar la experiencia del usuario, facilitando la generación automática de rutas y la obtención de información actualizada sobre los destinos turísticos.
	
	\item \textbf{Mejorar la accesibilidad a la información turística:} Proporcionar una plataforma fácil de usar que permita a los usuarios acceder rápidamente a descripciones, fotos y otros datos sobre los puntos de interés, mejorando su experiencia de exploración en las ciudades.
	
	\item \textbf{Rutas generadas sin intereses comerciales:} Generar rutas turísticas sin influencias comerciales, ofreciendo una experiencia imparcial y auténtica, en contraste con otras aplicaciones de recomendaciones de viajes.
\end{itemize}

\section{Catálogo de requisitos}
\subsection{Requisitos funcionales}
\begin{itemize}
	\item \textbf{RF-1 Solicitar permisos de uso de GPS:} La aplicación solicitará el permiso para acceder al GPS cuando se inicie por primera vez, ya que es necesario para calcular y mostrar la ubicación del usuario en tiempo real.
	
	\item \textbf{RF-2 Solicitud de activación de GPS:} Si el GPS está desactivado, la aplicación redirigirá a una pantalla que indicará al usuario la necesidad de activarlo para el correcto funcionamiento de la aplicación.
	
	\item \textbf{RF-3 Activación/Desactivación de seguimiento de usuario:} La aplicación mostrará en tiempo real el recorrido del usuario en el mapa, y este seguimiento podrá activarse o desactivarse en cualquier momento mediante un botón.
	
	\item \textbf{RF-4 Centrar la situación actual del usuario sobre el mapa:} El usuario podrá centrar manualmente su posición en el mapa mediante un botón dedicado. Además, existe la opción de fijar la ubicación del usuario en el centro del mapa durante su recorrido.
	
	\item \textbf{RF-5 Selección de Tour:} El usuario rellenará un formulario indicando el lugar que desea visitar, la cantidad de puntos de interés que quiere ver, sus preferencias de transporte (a pie o bicicleta), sus intereses, y el tiempo máximo que quiere dedicar a la ruta.
	
	\item \textbf{RF-6 Cálculo de información a través de un LLM:} La aplicación usará un servicio Gemini para generar los puntos de interés de acuerdo con las preferencias del usuario. Además, un servicio Google Places mejorará los datos proporcionando descripciones, fotos, URLs, ratings y número de votos de los POI.
	
	\item \textbf{RF-7 Eliminación de POI:} El usuario podrá eliminar puntos de interés tanto desde la pantalla del mapa como desde el resumen de la ruta. Cada vez que un POI es eliminado o añadido, la ruta se recalcula automáticamente para ofrecer el trayecto más óptimo.
	
	\item \textbf{RF-8 Cálculo de ruta optimizada:} La aplicación calculará la ruta más corta que conecte los puntos de interés seleccionados por el usuario, adaptándose al medio de transporte elegido (a pie o bicicleta).
	
	\item \textbf{RF-9 Capacidad de añadir un POI:} El usuario podrá agregar manualmente un lugar introduciendo su nombre en la barra de búsqueda. Si el lugar existe en los servicios de Google, será añadido automáticamente a la ruta; de lo contrario, no se tomará ninguna acción.
	
	\item \textbf{RF-10 Unirse a Eco City Tour:} El usuario podrá unirse a la ruta existente en cualquier momento. La aplicación calculará la ruta más corta para conectarlo con el tour, aunque esta ruta no se actualizará conforme avanza el tour.
	
	\item \textbf{RF-11 Mejora de los puntos de interés con servicio de obtención de información:} Los datos de los puntos de interés se enriquecerán con información adicional obtenida de Google Places, incluyendo ratings, imágenes, URL y número de votos, mejorando la experiencia del usuario.
\end{itemize}

\subsection{Requisitos no funcionales}
\begin{itemize}
	\item \textbf{RNF-1 Rendimiento:} la aplicación debe demostrar un tiempo de respuesta aceptable para que su manejo sea fluido y la carga de datos sea razonable al enlazar varios servicios asíncronos, de tal manera que no se perjudique la experiencia de usuario. 
	\item \textbf{RNF-2 Usabilidad:} Eco City Tours debe ser intuitiva y fácil de entender y utilizar.
	\item \textbf{RNF-3 Disponibilidad:} la aplicación debe estar disponible independientemente de la localización del usuario.
	\item \textbf{RNF-4 Mantenibilidad:} la aplicación debe ser fácilmente modificable debido a su caracter modular, permitiendo que sea fácil para el desarrollador su mantenimiento.
	\item \textbf{RNF-5 Escalabilidad:} Eco City Tours debe permitir la adición de nuevas funcionalidades que mejoren la experiencia de usuario de manera intuitiva.
	\item \textbf{RNF-6 Soporte:} la aplicación debe funcionar en versiones actuales de Android sin problemas de rendimiento o fallos en alguna de sus funcionalidades
\end{itemize}
\section{Especificación de requisitos}



\begin{table}[p]
	\centering
	\begin{tabularx}{\linewidth}{ p{0.21\columnwidth} p{0.71\columnwidth} }
		\toprule
		\textbf{CU-1}    & \textbf{Solicitar permisos de \acrshort{gps}.}\\
		\toprule
		\textbf{Versión}              & 1.0    \\
		\textbf{Autor}                & \autor \\
		\textbf{Requisitos asociados} & RF-1 \\
		\textbf{Descripción}          & Concede permisos de localización al dispositivo \\
		\textbf{Precondición}         & Acceder a la aplicación \\
		\textbf{Acciones}             &
		\begin{enumerate}
			\def\labelenumi{\arabic{enumi}.}
			\tightlist
			\item El usuario abre la aplicación o ha restringido los accesos a la localización
			\item Un mensaje del \acrfull{so} solicita conceder permisos.
			\item El usuario concede los permisos solicitados
		\end{enumerate}\\
		\textbf{Postcondición}        & El permiso de seguimiento del usuario ha sido concedido \\
		\textbf{Excepciones}          & El usuario no concede los permisos \\
		\textbf{Importancia}          & Alta  \\
		\bottomrule
	\end{tabularx}
	\caption{CU-1 Solicitar permisos de \acrshort{gps}.}
\end{table}

\begin{table}[p]
	\centering
	\begin{tabularx}{\linewidth}{ p{0.21\columnwidth} p{0.71\columnwidth} }
		\toprule
		\textbf{CU-1}    & \textbf{Activar uso de sensor \acrshort{gps} en el dispositivo}\\
		\toprule
		\textbf{Versión}              & 1.0    \\
		\textbf{Autor}                & \autor \\
		\textbf{Requisitos asociados} & RF-2 \\
		\textbf{Descripción}          & Activa el uso de GPS en el dispositivo \\
		\textbf{Precondición}         & El dispositivo debe tener un sensor \acrshort{gps}. \\
		\textbf{Acciones}             &
		\begin{enumerate}
			\def\labelenumi{\arabic{enumi}.}
			\tightlist
			\item El usuario activará el sensor de \acrshort{gps} de su dispositivo a través de cualquier menú o atajo de configuración.
		\end{enumerate}\\
		\textbf{Postcondición}        & La aplicación tiene acceso a la posición del dispositivo. \\
		\textbf{Excepciones}          & Problemas con el sensor \acrshort{gps} del dispositivo. \\
		\textbf{Importancia}          & Alta \\
		\bottomrule
	\end{tabularx}
	\caption{CU-2 Activación de sensor \acrshort{gps} en el dispositivo.}
\end{table}


\begin{table}[p]
	\centering
	\begin{tabularx}{\linewidth}{ p{0.21\columnwidth} p{0.71\columnwidth} }
		\toprule
		\textbf{CU-1}    & \textbf{Eliminar un \acrlong{pdi}}\\
		\toprule
		\textbf{Versión}              & 1.0    \\
		\textbf{Autor}                & \autor \\
		\textbf{Requisitos asociados} & RF-1, RF-2, RF-5, RF-6, RF-8 \\
		\textbf{Descripción}          & Genera una ruta turística en función de las preferencias del usuario \\
		\textbf{Precondición}         & El dispositivo debe tener acceso a los permisos de localización y estar usando el sensor \acrshort{gps} \\
		\textbf{Acciones}             &
		\begin{enumerate}
			\def\labelenumi{\arabic{enumi}.}
			\tightlist
			\item El usuario activará introducirá destino en la interfaz.
			\item Elegirá cuantos sitios quiere visitar.
			\item Indicará sus gustos a la hora de viajar: naturaleza, museos...
			\item Indicará el tiempo máximo que quiere invertir en el trayecto.
			\item Se acciona el botón de obtención de ruta.
			\item El sistema muestra la ruta generada sobre un mapa.
		\end{enumerate}\\
		\textbf{Postcondición}        & La información resultante es mostrada en un mapa. \\
		\textbf{Excepciones}          & Alguno de los servicios generadores de ruta no devuelve el resultado esperado. \\
		\textbf{Importancia}          & Alta \\
		\bottomrule
	\end{tabularx}
	\caption{CU-3 Generación de un Eco City Tour.}
\end{table}



\begin{table}[p]
	\centering
	\begin{tabularx}{\linewidth}{ p{0.21\columnwidth} p{0.71\columnwidth} }
		\toprule
		\textbf{CU-1}    & \textbf{Ejemplo de caso de uso}\\
		\toprule
		\textbf{Versión}              & 1.0    \\
		\textbf{Autor}                & Alumno \\
		\textbf{Requisitos asociados} & RF-xx, RF-xx \\
		\textbf{Descripción}          & La descripción del CU \\
		\textbf{Precondición}         & Precondiciones (podría haber más de una) \\
		\textbf{Acciones}             &
		\begin{enumerate}
			\def\labelenumi{\arabic{enumi}.}
			\tightlist
			\item Pasos del CU
			\item Pasos del CU (añadir tantos como sean necesarios)
		\end{enumerate}\\
		\textbf{Postcondición}        & Postcondiciones (podría haber más de una) \\
		\textbf{Excepciones}          & Excepciones \\
		\textbf{Importancia}          & Alta o Media o Baja... \\
		\bottomrule
	\end{tabularx}
	\caption{CU-1 Nombre del caso de uso.}
\end{table}
