\apendice{Plan de Proyecto Software}

\section{Introducción}
El Plan de Proyecto Software es el documento clave que dirige el proceso de desarrollo de la aplicación móvil creada. Este apéndice tiene como objetivo detallar los aspectos críticos de la planificación y gestión del proyecto, asegurando una implementación eficiente y efectiva.
La planificación temporal del proyecto se ha llevado a cabo con el \emph{uso de la metodología ágil} buscando dividir el desarollo en tareas, sprints e hitos que producen un resultado iterativos y bien estructurado, lo que conlleva a una mayor flexibilidad y a la adaptación más efectiva frente los cambios. 

A continuación, se determinará la viabilidad, que reflejará los \emph{recursos humanos y materiales}, así como los costes asociados necesarios para su valoración. La viabilidad incluirá una estimación de los fondos que se basarán en el salario de un trabajador de la imaginación, así como un análisis de los posibles riesgos y su mitigación. Los aspectos económicos y técnicos de la viabilidad son fundamentales, ya que de ellos depende de que el proyecto esté en los límites establecidos y cumpla con los objetivos propuestos.

Este plan es esencial para la gestión del proyecto ya que sirve como una guía detallada, ayudando así a identificar y mitigar riesgos así como a la utilización eficaz de los recursos. Con el enfoque estructurado y ágil, proporcionado por este plan, el equipo de desarrollo podría entregar un producto de alta calidad que, además, cumplirá con el nivel de satisfacción alcanzado entre los usuarios o clientes. 
\section{Planificación temporal}
 Como se ha mencionado anteriormente, la planificación temporal del proyecto se ha llevado a cabo con el uso de metodología ágil: se basa en la división del desarrollo en tareas, sprints e hitos que producen un resultado iterativo y bien estructurado. Esto conlleva una mayor flexibilidad y a la adaptación más efectiva frente a los cambios.

 Algunas herramientas utilizadas para la planificación temporal han sido GitHub y Zube. Ésta última ha permitido la organización de las tareas en tableros Kanban. El uso de GitHub ha permitido gestionar un control de versiones.

 A continuación veremos como la planificación temporal se ha llevado a cabo en diferentes sprints, cómo se ha ido iterando en las diferentes partes del proyecto y cómo se han ido cumpliendo los hitos propuestos. Para ello se mostrarán diferentes diagramas basadas en métricas ágiles.

 Cada tarea se ha dividido en diferentes historias de usuario, que se han ido completando en cada sprint. Cada sprint ha tenido una duración de una o dos semanas, y se han ido completando las tareas propuestas en cada uno de ellos.
 Un ejemplo se puede observar en la siguiente figura ~\ref{fig:issue12}
 \imagen{issue12}{Tarea 12 mostrada en GitHub con la descripción, hito y etiquetas de la tarea a realizar.}

Gracias al uso de la herramienta Zube, se ha podido llevar un control de las tareas a realizar, las tareas completadas y las tareas pendientes. Además, se ha podido llevar un control de los hitos propuestos y de las historias de usuario completadas en cada sprint. Un ejemplo de ello se puede observar en la siguiente figura ~\ref{fig:sprint1}
\imagen{sprint1}{Tablero Kanban de Zube con la gestión de tareas del Sprint 1.}

\subsection{\textit{Hitos}}
Los hitos o milestones son puntos de referencia que marcan el final de un conjunto de tareas. En este proyecto se han definido los siguientes hitos:

\begin{itemize}

    \item \textbf{Kick-off} Puesta en marcha del proyecto. A partir de las reuniones mantenidas con el tutor, se necesita tener todas las herramientas preparadas para empezar a desarrollar tanto la aplicación como su documentación.
    
    \item \textbf{MPV - Mínimo Producto Viable} Se define el MVP como una aplicación móvil que sobre un mapa OSM muestre la ubicación de usuario, obtenga unos POI básicos y una ruta que las una.

    \item \textbf{Checkpoint 1 de documentación} Este milestone agrupa las tareas relacionadas con la creación y actualización de la documentación del proyecto hasta la reunión con el tutor el 1 de septiembre de 2024.

    El objetivo es tener una documentación suficiente para que el tutor pueda dar feedback acerca de la misma y poder corregir errores.

    \item \textbf{Prototipo Prompting} Este prototipo se puede realizar en un cuaderno Jupyter y su objetivo es mostrar la evolución en el prompt que dará como resultado unos POI de mayor calidad.
    
    \item \textbf{Prototipo con tours generados por LLM} El objetivo es transitar desde una aplicación inicial capaz de mostrar lugares y rutas en un mapa, hacia una aplicación que sea capaz de conseguir que estos mismos marcadores y polilíneas sean generados a través de un LLM. \label{hito:prototipo_llm}
\end{itemize}

\subsection{Organización en \textit{Sprints}}

Al comenzar este proyecto durante periodo no lectivo se realizaron los Sprint con variación de tiempo de una o dos semanas en función de la planificación personal. Una vez comenzado el curso y con la ayuda del tutor se realizaron reuniones que han servido para, siguiendo la metodología \textit{Agile}, revisar el Sprint anterior, planificar el siguiente y hacer una pequeña retrospectiva para mejorar el trabajo conjunto.


\begin{itemize}
    \item \textbf{Sprint Kick-off(22/07/2024 - 29/07/2024):} Después de las reuniones con el tutor, se establecen los objetivos del proyecto y se comienza a trabajar en la puesta en marcha del proyecto. Se establecen las herramientas a utilizar y se comienza a trabajar en la documentación del proyecto. 33 puntos de historia en 5 tareas.
    ~\ref{fig:bd-kick-off sprint}
	\imagen{bd-kick-off sprint}{Figura burndown del Sprint Kick-off.}
    

    \item \textbf{Sprint 1(29/07/2024 - 05/08/2024):} Con las herramientas y una idea previa establecida, es el momento de empezar a desarrollar.

    Objetivos: seguir formándome en LLM y las opciones que pueda implementar en el prototipo de prompt.
    Empezar a desarrollar la aplicación móvil con las características básicas.
    Aprender a documentar sprints, indicar elementos que tendré que documentar y aquellos que tenga claro ir documentando para hacer un avance significativo que pueda evaluar mi tutor.
    
    \item \textbf{Sprint 2(05/08/2024 - 12/08/2024):} A partir del concepto básico, se añaden pequeñas mejoras en los tres aspectos del proyecto.
    
    Objetivos: mejorar el prototipo de prompting del cuaderno Jupyter hasta incorporar un sistema RAG, incluir marcadores al mapa en cuanto al desarrollo y continuar con la documentación.
    
    \textit{Dificultades encontradas}: la documentación me hizo perder mucho tiempo debido a problemas con las librerías, después de mucho tiempo reinicié el proyecto desde la plantilla dada, insertando el texto, lo que solucionó el problema. En cuanto al diseño de la aplicación, el desarrollo fue lento al tener que evaluar diferentes opciones ya que la mayoría de fuentes utilizan mapas de Google, opción que se quería descartar.

    \item \textbf{Sprint 3(12/08/2024 - 22/08/2024):} Este sprint fue más largo que los anteriores para mejorar el resultado final ya que la intención era dejar el proyecto en un estado de revisión lo más completo posible para afrontar la reunión prevista para inicio de septiembre con el tutor del mismo. Al intentar desarrollar la tecnología de enrutado del usuario se comprendió lo que ya se intuía en el sprint anterior y es que basar el trabajo en servicios de Google iba a reportar en un desarrollo más fácil y un resultado más robusto y fiable como se justifica en la sección 5 de la memoria de este \acrshort{tfg}.
    
    \item \textbf{Sprint 4(22/08/2024 - 02/09/2024):} El objetivo es mostrar la versión más completa de la aplicación, la documentación y ahondar en el uso de nuevas herramientas como Figma como herramienta de diseño de aplicaciones y LangFlow a la hora de utilizar otro modelo de prototipo.
    
\item \textbf{S5 - Aplicación con origen de datos LLM (Preparación y documentación):} Después de la reunión con el tutor de inicio de septiembre y habiendo cumplido los objetivos de los primeros hitos se decide continuar intentando alcanzar el hito \ref{hito:prototipo_llm}. Para ello se prepara y documenta primero en este sprint el desarrollo necesario.

\item \textbf{S6 - Desarrollo y Finalización prototipo google\_generative\_ai como LLM} Habiendo encontrado una solución óptima al modelo LLM a utilizar se propone realizar un prototipo que implemente la interfaz de usuario y su conexión con el modelo LLM. En los diferentes apartados encontramos. Se trata de un sprint de prominente desarrollo de la aplicación. Se consiguió reestructurar todo el código centralizando labores de gestión del tour generado y sus \acrshort{poi} manteniendo la modularidad del código.
	
	\item \textbf{S7 - Consolidación y Calidad} Habiendo cumplido el hito \ref{hito:prototipo_llm} y teniendo una aplicación con muchas funcionalidades buscadas, durante este sprint se busca consolidar el código y dotarlo de una calidad y mantenimiento con herramientas de soporte como Sonar Cloud o Logger.
    
\end{itemize}

\subsection{Métricas Ágiles}
Gracias al uso de Zube se hace uso de diferentes métricas ágiles que han sido vitales para la evaluación del desarrollo de la aplicación y su organización a través de los sprints citados. Algunos de los artefactos usados son los siguientes:
\subsubsection{Gráficos burnup / burndown} 
Muestran a lo largo del tiempo de un sprint la evolución de tareas realizadas por el equipo de desarrollo.
\imagen{burnup-s6}{Gráfico Burnup del Sprint 6}{1}

\section{Estudio de viabilidad}

\subsection{Viabilidad económica}

\subsection{Viabilidad legal}


