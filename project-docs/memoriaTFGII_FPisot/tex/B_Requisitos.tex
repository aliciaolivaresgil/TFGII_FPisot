\apendice{Especificación de Requisitos}

\section{Introducción}
En esta sección se presentan los requisitos de la aplicación, abordando tanto los objetivos generales como los específicos del proyecto. Se incluye un catálogo detallado de los requisitos funcionales y no funcionales, que definen el comportamiento y las características técnicas de la aplicación. Además, se proporciona una especificación detallada de los requisitos a través de tablas de casos de uso, complementadas con su respectivo diagrama de casos de uso, lo que facilita una comprensión clara de las interacciones principales de los usuarios con el sistema.


\section{Objetivos generales}
La misión fundamental de este proyecto persigue conseguir los siguientes propósitos:
\begin{itemize}
	\item \textbf{Fomentar el turismo sostenible:} Facilitar a los usuarios la exploración de ciudades promoviendo al mostrar rutas no motorizadas y modos de transporte como caminar y el uso de bicicletas.
	
	\item \textbf{Optimización de experiencias turísticas personalizadas:} Ofrecer a los usuarios rutas personalizadas que se ajusten a sus intereses y preferencias, proporcionando información detallada y relevante sobre los puntos de interés seleccionados.
	
	\item \textbf{Promover el uso de tecnologías inteligentes en el turismo:} Utilizar tecnologías avanzadas como servicios GIS, Google Places y LLMs para mejorar la experiencia del usuario, facilitando la generación automática de rutas y la obtención de información actualizada sobre los destinos turísticos.
	
	\item \textbf{Mejorar la accesibilidad a la información turística:} Proporcionar una plataforma fácil de usar que permita a los usuarios acceder rápidamente a descripciones, fotos y otros datos sobre los puntos de interés, mejorando su experiencia de exploración en las ciudades.
	
	\item \textbf{Rutas generadas sin intereses comerciales:} Generar rutas turísticas sin influencias comerciales, ofreciendo una experiencia imparcial y auténtica, en contraste con otras aplicaciones de recomendaciones de viajes.
\end{itemize}

\section{Catálogo de requisitos}
\subsection{Requisitos funcionales}
\begin{itemize}
	\item \textbf{RF-1 Solicitar permisos de uso de GPS:} La aplicación solicitará el permiso para acceder al GPS cuando se inicie por primera vez, ya que es necesario para calcular y mostrar la ubicación del usuario en tiempo real.
	
	\item \textbf{RF-2 Solicitud de activación de GPS:} Si el GPS está desactivado, la aplicación redirigirá a una pantalla que indicará al usuario la necesidad de activarlo para el correcto funcionamiento de la aplicación.
	
	\item \textbf{RF-3 Activación/Desactivación de seguimiento de usuario:} La aplicación mostrará en tiempo real el recorrido del usuario en el mapa, y este seguimiento podrá activarse o desactivarse en cualquier momento mediante un botón.
	
	\item \textbf{RF-4 Centrar la situación actual del usuario sobre el mapa:} El usuario podrá centrar manualmente su posición en el mapa mediante un botón dedicado. Además, existe la opción de fijar la ubicación del usuario en el centro del mapa durante su recorrido.
	
	\item \textbf{RF-5 Selección de Tour:} El usuario rellenará un formulario indicando el lugar que desea visitar, la cantidad de puntos de interés que quiere ver, sus preferencias de transporte (a pie o bicicleta), sus intereses, y el tiempo máximo que quiere dedicar a la ruta.
	
	\item \textbf{RF-6 Cálculo de información a través de un LLM:} La aplicación usará un servicio Gemini para generar los puntos de interés de acuerdo con las preferencias del usuario. Además, un servicio Google Places mejorará los datos proporcionando descripciones, fotos, URLs, ratings y número de votos de los POI.
	
	\item \textbf{RF-7 Eliminación de POI:} El usuario podrá eliminar puntos de interés tanto desde la pantalla del mapa como desde el resumen de la ruta. Cada vez que un POI es eliminado o añadido, la ruta se recalcula automáticamente para ofrecer el trayecto más óptimo.
	
	\item \textbf{RF-8 Cálculo de ruta optimizada:} La aplicación calculará la ruta más corta que conecte los puntos de interés seleccionados por el usuario, adaptándose al medio de transporte elegido (a pie o bicicleta).
	
	\item \textbf{RF-9 Capacidad de añadir un POI:} El usuario podrá agregar manualmente un lugar introduciendo su nombre en la barra de búsqueda. Si el lugar existe en los servicios de Google, será añadido automáticamente a la ruta; de lo contrario, no se tomará ninguna acción.
	
	\item \textbf{RF-10 Unirse a Eco City Tour:} El usuario podrá unirse a la ruta existente en cualquier momento. La aplicación calculará la ruta más corta para conectarlo con el tour, aunque esta ruta no se actualizará conforme avanza el tour.
	
	\item \textbf{RF-11 Mejora de los puntos de interés con servicio de obtención de información:} Los datos de los puntos de interés se enriquecerán con información adicional obtenida de Google Places, incluyendo ratings, imágenes, URL y número de votos, mejorando la experiencia del usuario.
\end{itemize}

\subsection{Requisitos no funcionales}
\begin{itemize}
	\item \textbf{RNF-1 Rendimiento:} la aplicación debe demostrar un tiempo de respuesta aceptable para que su manejo sea fluido y la carga de datos sea razonable al enlazar varios servicios asíncronos, de tal manera que no se perjudique la experiencia de usuario. 
	\item \textbf{RNF-2 Usabilidad:} Eco City Tours debe ser intuitiva y fácil de entender y utilizar.
	\item \textbf{RNF-3 Disponibilidad:} la aplicación debe estar disponible independientemente de la localización del usuario.
	\item \textbf{RNF-4 Mantenibilidad:} la aplicación debe ser fácilmente modificable debido a su caracter modular, permitiendo que sea fácil para el desarrollador su mantenimiento.
	\item \textbf{RNF-5 Escalabilidad:} Eco City Tours debe permitir la adición de nuevas funcionalidades que mejoren la experiencia de usuario de manera intuitiva.
	\item \textbf{RNF-6 Soporte:} la aplicación debe funcionar en versiones actuales de Android sin problemas de rendimiento o fallos en alguna de sus funcionalidades
\end{itemize}
\clearpage
\section{Especificación de requisitos}
\imagen{diagrama-casos-de-uso}{Diagrama de casos de uso}


\begin{table}[p]
	\centering
	\begin{tabularx}{\linewidth}{ p{0.21\columnwidth} p{0.71\columnwidth} }
		\toprule
		\textbf{CU-1}    & \textbf{Solicitar permisos de \acrshort{gps}.}\\
		\toprule
		\textbf{Versión}              & 1.0    \\
		\textbf{Autor}                & \autor \\
		\textbf{Requisitos asociados} & RF-1 \\
		\textbf{Descripción}          & Concede permisos de localización al dispositivo \\
		\textbf{Precondición}         & Acceder a la aplicación \\
		\textbf{Acciones}             &
		\begin{enumerate}
			\def\labelenumi{\arabic{enumi}.}
			\tightlist
			\item El usuario abre la aplicación o ha restringido los accesos a la localización
			\item Un mensaje del \acrfull{so} solicita conceder permisos.
			\item El usuario concede los permisos solicitados
		\end{enumerate}\\
		\textbf{Postcondición}        & El permiso de seguimiento del usuario ha sido concedido. Si no es concedido, la aplicación no podrá generar rutas personalizadas basadas en la ubicación. \\
		\textbf{Excepciones}          & El usuario no concede los permisos. \\
		\textbf{Importancia}          & Alta  \\
		\bottomrule
	\end{tabularx}
	\caption{CU-1 Solicitar permisos de \acrshort{gps}.}
\end{table}

\begin{table}[p]
	\centering
	\begin{tabularx}{\linewidth}{ p{0.21\columnwidth} p{0.71\columnwidth} }
		\toprule
		\textbf{CU-1}    & \textbf{Activar sensor \acrshort{gps}.}\\
		\toprule
		\textbf{Versión}              & 1.0    \\
		\textbf{Autor}                & \autor \\
		\textbf{Requisitos asociados} & RF-2 \\
		\textbf{Descripción}          & Activa el uso de GPS en el dispositivo \\
		\textbf{Precondición}         & El dispositivo debe tener un sensor \acrshort{gps}. \\
		\textbf{Acciones}             &
		\begin{enumerate}
			\def\labelenumi{\arabic{enumi}.}
			\tightlist
			\item \item La aplicación detecta que el GPS está desactivado y notifica al usuario.
			\item El usuario activará el sensor de \acrshort{gps} de su dispositivo a través de cualquier menú o atajo de configuración.
		\end{enumerate}\\
		\textbf{Postcondición}        & La aplicación tiene acceso a la posición del dispositivo. \\
		\textbf{Excepciones}          & Problemas con el sensor \acrshort{gps} del dispositivo. \\
		\textbf{Importancia}          & Alta \\
		\bottomrule
	\end{tabularx}
	\caption{CU-2 Activar sensor \acrshort{gps}.}
\end{table}

\begin{table}[p]
	\centering
	\begin{tabularx}{\linewidth}{ p{0.21\columnwidth} p{0.71\columnwidth} }
		\toprule
		\textbf{CU-3}    & \textbf{Activar/Desactivar el seguimiento del usuario.}\\
		\toprule
		\textbf{Versión}              & 1.0    \\
		\textbf{Autor}                & \autor \\
		\textbf{Requisitos asociados} & RF-3 \\
		\textbf{Descripción}          & El usuario puede activar o desactivar el seguimiento de su posición en tiempo real en el mapa, permitiendo ver el trayecto mientras se desplaza. \\
		\textbf{Precondición}         & La aplicación tiene acceso a la ubicación del usuario y tiene permisos de su uso. \\
		\textbf{Acciones}             &
		\begin{enumerate}
			\def\labelenumi{\arabic{enumi}.}
			\tightlist
			\item El usuario selecciona la opción de activar o desactivar el seguimiento de su ubicación en el mapa.
			\item El sistema ajusta el mapa para mostrar o dejar de mostrar el movimiento del usuario en tiempo real.
		\end{enumerate}\\
		\textbf{Postcondición}        & El mapa sigue o deja de seguir la posición del usuario en tiempo real. \\
		\textbf{Excepciones}          & Problemas de conexión con el GPS o pérdida de señal. \\
		\textbf{Importancia}          & Media  \\
		\bottomrule
	\end{tabularx}
	\caption{CU-3 Activar/Desactivar el seguimiento del usuario.}
\end{table}

\begin{table}[p]
	\centering
	\begin{tabularx}{\linewidth}{ p{0.21\columnwidth} p{0.71\columnwidth} }
		\toprule
		\textbf{CU-4}    & \textbf{Centrar mapa en la ubicación actual.}\\
		\toprule
		\textbf{Versión}              & 1.0    \\
		\textbf{Autor}                & \autor \\
		\textbf{Requisitos asociados} & RF-4 \\
		\textbf{Descripción}          & El usuario puede centrar manualmente la vista del mapa en su ubicación actual mediante un botón. \\
		\textbf{Precondición}         & La aplicación tiene acceso a la ubicación del usuario y tiene permisos de su uso. \\
		\textbf{Acciones}             &
		\begin{enumerate}
			\def\labelenumi{\arabic{enumi}.}
			\tightlist
			\item El usuario pulsa el botón para centrar el mapa en su ubicación actual.
			\item La aplicación ajusta el mapa para centrar la vista en la posición del usuario.
		\end{enumerate}\\
		\textbf{Postcondición}        & El mapa se centra en la ubicación actual del usuario. \\
		\textbf{Excepciones}          & La ubicación del usuario no está disponible o la señal GPS es débil. \\
		\textbf{Importancia}          & Baja \\
		\bottomrule
	\end{tabularx}
	\caption{CU-4 Centrar mapa en la ubicación actual.}
\end{table}

\begin{table}[p]
	\centering
	\begin{tabularx}{\linewidth}{ p{0.21\columnwidth} p{0.71\columnwidth} }
		\toprule
		\textbf{CU-5}    & \textbf{Completar formulario de preferencias.}\\
		\toprule
		\textbf{Versión}              & 1.0    \\
		\textbf{Autor}                & \autor \\
		\textbf{Requisitos asociados} & RF-5 \\
		\textbf{Descripción}          & El usuario rellena un formulario indicando el lugar que quiere visitar, el número de puntos de interés, el medio de transporte, sus intereses y el tiempo que quiere dedicar a la ruta. \\
		\textbf{Precondición}         & La aplicación debe estar activa y mostrar el formulario. \\
		\textbf{Acciones}             &
		\begin{enumerate}
			\def\labelenumi{\arabic{enumi}.}
			\tightlist
			\item El usuario selecciona el lugar de destino.
			\item El usuario elige el número de puntos de interés que desea visitar.
			\item El usuario selecciona el modo de transporte (a pie o en bicicleta).
			\item El usuario selecciona sus intereses (naturaleza, historia, deportes, etc.).
			\item El usuario indica el tiempo máximo que quiere dedicar a la ruta.
		\end{enumerate}\\
		\textbf{Postcondición}        & El formulario queda completado y la aplicación está lista para generar una ruta personalizada. \\
		\textbf{Excepciones}          & El usuario no completa el formulario o deja campos vacíos. \\
		\textbf{Importancia}          & Alta  \\
		\bottomrule
	\end{tabularx}
	\caption{CU-5 Completar formulario de preferencias.}
\end{table}

\begin{table}[p]
	\centering
	\begin{tabularx}{\linewidth}{ p{0.21\columnwidth} p{0.71\columnwidth} }
		\toprule
		\textbf{CU-6}    & \textbf{Generar ruta optimizada.}\\
		\toprule
		\textbf{Versión}              & 1.0    \\
		\textbf{Autor}                & \autor \\
		\textbf{Requisitos asociados} & RF-5, RF-6, RF-8 \\
		\textbf{Descripción}          & El sistema genera una ruta optimizada que conecta los puntos de interés basados en las preferencias del usuario, consultando servicios externos como LLM, Google Places, y un servicio de optimización de rutas. \\
		\textbf{Precondición}         & El formulario de preferencias ha sido completado y la aplicación tiene acceso a los servicios externos. \\
		\textbf{Acciones}             &
		\begin{enumerate}
			\def\labelenumi{\arabic{enumi}.}
			\tightlist
			\item El usuario envía las preferencias desde el formulario (lugar, número de puntos de interés, transporte, tiempo, etc.).
			\item El sistema consulta un servicio LLM para obtener los puntos de interés (POI) basados en las preferencias del usuario.
			\item El sistema envía los POI al servicio Google Places para obtener información mejorada (descripciones, fotos, ratings, etc.).
			\item El sistema envía la información GPS de los POI a un servicio de optimización de rutas para calcular la ruta más corta.
			\item El sistema muestra en el mapa la ruta optimizada, conectando los POI según las preferencias del usuario.
		\end{enumerate}\\
		\textbf{Postcondición}        & La ruta optimizada es mostrada en el mapa, incluyendo la información mejorada de los POI. \\
		\textbf{Excepciones}          & Problemas en la conexión con los servicios LLM, Google Places o el servicio de optimización de rutas. \\
		\textbf{Importancia}          & Alta \\
		\bottomrule
	\end{tabularx}
	\caption{CU-6 Generar ruta optimizada.}
\end{table}



\begin{table}[p]
	\centering
	\begin{tabularx}{\linewidth}{ p{0.21\columnwidth} p{0.71\columnwidth} }
		\toprule
		\textbf{CU-7}    & \textbf{Visualizar detalles de puntos de interés.}\\
		\toprule
		\textbf{Versión}              & 1.0    \\
		\textbf{Autor}                & \autor \\
		\textbf{Requisitos asociados} & RF-6, RF-11 \\
		\textbf{Descripción}          & El usuario puede hacer tap en un marcador del mapa para obtener información detallada sobre un punto de interés, como su descripción, fotos, ratings y URL. \\
		\textbf{Precondición}         & La ruta ha sido generada y los puntos de interés están visibles en el mapa. \\
		\textbf{Acciones}             &
		\begin{enumerate}
			\def\labelenumi{\arabic{enumi}.}
			\tightlist
			\item El usuario selecciona un marcador de punto de interés en el mapa.
			\item El sistema muestra la información detallada del punto de interés seleccionado.
		\end{enumerate}\\
		\textbf{Postcondición}        & Se muestra la información detallada del punto de interés seleccionado. \\
		\textbf{Excepciones}          & El sistema no puede obtener detalles del punto de interés seleccionado debido a problemas con los servicios externos. \\
		\textbf{Importancia}          & Media \\
		\bottomrule
	\end{tabularx}
	\caption{CU-7 Visualizar detalles de puntos de interés.}
\end{table}

\begin{table}[p]
	\centering
	\begin{tabularx}{\linewidth}{ p{0.21\columnwidth} p{0.71\columnwidth} }
		\toprule
		\textbf{CU-8}    & \textbf{Eliminar puntos de interés.}\\
		\toprule
		\textbf{Versión}              & 1.0    \\
		\textbf{Autor}                & \autor \\
		\textbf{Requisitos asociados} & RF-7 \\
		\textbf{Descripción}          & El usuario puede eliminar puntos de interés desde el mapa o la pantalla de resumen, lo que recalcula automáticamente la ruta optimizada. \\
		\textbf{Precondición}         & La ruta ha sido generada y los puntos de interés están visibles. \\
		\textbf{Acciones}             &
		\begin{enumerate}
			\def\labelenumi{\arabic{enumi}.}
			\tightlist
			\item El usuario selecciona un punto de interés para eliminarlo.
			\item El sistema elimina el punto de interés seleccionado de la ruta.
			\item El sistema recalcula la ruta optimizada con los puntos de interés restantes.
		\end{enumerate}\\
		\textbf{Postcondición}        & La ruta es recalculada sin el punto de interés eliminado. \\
		\textbf{Excepciones}          & Problemas en la recalculación de la ruta o eliminación del punto de interés. \\
		\textbf{Importancia}          & Media  \\
		\bottomrule
	\end{tabularx}
	\caption{CU-8 Eliminar puntos de interés.}
\end{table}

\begin{table}[p]
	\centering
	\begin{tabularx}{\linewidth}{ p{0.21\columnwidth} p{0.71\columnwidth} }
		\toprule
		\textbf{CU-9}    & \textbf{Añadir puntos de interés a la ruta.}\\
		\toprule
		\textbf{Versión}              & 1.0    \\
		\textbf{Autor}                & \autor \\
		\textbf{Requisitos asociados} & RF-9 \\
		\textbf{Descripción}          & El usuario puede añadir manualmente nuevos lugares a la ruta existente introduciendo el nombre del lugar en la barra de búsqueda en el mapa. \\
		\textbf{Precondición}         & La ruta debe haber sido generada previamente. \\
		\textbf{Acciones}             &
		\begin{enumerate}
			\def\labelenumi{\arabic{enumi}.}
			\tightlist
			\item El usuario introduce el nombre de un lugar en la barra de búsqueda.
			\item El sistema valida si el lugar existe y lo añade a la ruta.
			\item El sistema recalcula la ruta optimizada incluyendo el nuevo punto de interés.
		\end{enumerate}\\
		\textbf{Postcondición}        & El nuevo lugar es añadido a la ruta y la ruta optimizada es recalculada. \\
		\textbf{Excepciones}          & El lugar no se encuentra en los servicios de búsqueda o hay problemas al recalcular la ruta. \\
		\textbf{Importancia}          & Media \\
		\bottomrule
	\end{tabularx}
	\caption{CU-9 Añadir puntos de interés a la ruta.}
\end{table}

\begin{table}[p]
	\centering
	\begin{tabularx}{\linewidth}{ p{0.21\columnwidth} p{0.71\columnwidth} }
		\toprule
		\textbf{CU-10}    & \textbf{Unirse a la ruta calculada.}\\
		\toprule
		\textbf{Versión}              & 1.0    \\
		\textbf{Autor}                & \autor \\
		\textbf{Requisitos asociados} & RF-10 \\
		\textbf{Descripción}          & El sistema calcula la ruta más corta para que el usuario se una a una ruta previamente generada, partiendo de su ubicación actual. \\
		\textbf{Precondición}         & Una ruta ya ha sido generada y está activa. \\
		\textbf{Acciones}             &
		\begin{enumerate}
			\def\labelenumi{\arabic{enumi}.}
			\tightlist
			\item El usuario solicita unirse a la ruta calculada desde su ubicación actual.
			\item El sistema calcula la ruta más corta para conectar la ubicación actual del usuario con la ruta generada.
		\end{enumerate}\\
		\textbf{Postcondición}        & El usuario es guiado desde su ubicación actual hasta la ruta generada. \\
		\textbf{Excepciones}          & Problemas con los servicios de generación de rutas. \\
		\textbf{Importancia}          & Media \\
		\bottomrule
	\end{tabularx}
	\caption{CU-10 Unirse a la ruta calculada.}
\end{table}

\begin{table}[p]
	\centering
	\begin{tabularx}{\linewidth}{ p{0.21\columnwidth} p{0.71\columnwidth} }
		\toprule
		\textbf{CU-11}    & \textbf{Visualizar resumen de la ruta.}\\
		\toprule
		\textbf{Versión}              & 1.0    \\
		\textbf{Autor}                & \autor \\
		\textbf{Requisitos asociados} & RF-11 \\
		\textbf{Descripción}          & El usuario puede visualizar un resumen de la ruta generada, incluyendo la distancia total, duración estimada y el medio de transporte seleccionado. \\
		\textbf{Precondición}         & La ruta ha sido generada. \\
		\textbf{Acciones}             &
		\begin{enumerate}
			\def\labelenumi{\arabic{enumi}.}
			\tightlist
			\item El usuario accede a la pantalla de resumen de la ruta.
			\item El sistema muestra el resumen de la ruta con la distancia total, duración estimada y el medio de transporte.
		\end{enumerate}\\
		\textbf{Postcondición}        & El resumen de la ruta es visible para el usuario. \\
		\textbf{Excepciones}          & Problemas para acceder a la información de la ruta o problemas de conectividad con los servicios externos. \\
		\textbf{Importancia}          & Baja \\
		\bottomrule
	\end{tabularx}
	\caption{CU-11 Visualizar resumen de la ruta.}
\end{table}
